%************************************************
\section{Classe Unit Testing}
\label{class:unittest}
%************************************************

Unit testing è un approccio allo sviluppo del software in cui vengono scritti test per ogni funzione nell'applicazione. 

%Se non si ha familiarità con il concetto si potrebbe fare un po 'googling sull'argomento. 

La classe che risulta abbastanza intuitiva, è costituita da una funzione di valutazione e da due funzioni di risultato. Non deve essere considerata come uno strumento per fare veri e propri test di debug, ma piuttosto come un semplice meccanismo per valutare il codice e determinare se si sta producendo il tipo di dati e il risultato corretti.

La classe viene inizializzata nel proprio controller utilizzando la funzione \var{\$this->load->library}:

\begin{code}
$this->load->library('unit_test');
\end{code}

Una volta caricata la classe, l'oggetto Unit Test sarà disponibile utilizzando \var{\$this->unit}.

\section*{Test}

Per eseguire un test è necessario che si fornisca una condizione e il risultato atteso tramite la seguente funzione:

\begin{itemize}
\item \verb|$this->unit->run( test, expected result, 'test name', 'notes')|
\end{itemize}

Dove \var{test} è il risultato del codice che si vuole testare, \var{expected result} è il dato che ci si aspetta \var{test name} è un nome opzionale che può essere dato al test, e \var{notes} è un commento, anch'esso opzionale. Per esempio:

\begin{code}
$test = 1 + 1;

$expected_result = 2;

$test_name = 'Somma di uno piu' uno';

$this->unit->run($test, $expected_result, $test_name);
\end{code}

Il risultato atteso che si fornisce alla funzione deve avere una corrispondenza ``letterale'' o per ``tipo''. Ecco un esempio di corrispondenza letterale:

\begin{code}
$this->unit->run('Foo', 'Foo');
\end{code}

Ecco un esempio di corrispondenza per tipo:

\begin{code}
$this->unit->run('Foo', 'is_string');
\end{code}

In quest'ultimo esempio, come si sarà intuito, il secondo parametro indica alla funzione di valutare se il tipo di dato è una stringa. Ecco una lista dei tipi di dati utilizzabili nel test:

\begin{code}
is_object
is_string
is_bool
is_true
is_false
is_int
is_numeric
is_float
is_double
is_array
is_null
\end{code}

\section*{Generare un Report}

\'E possibile visualizzare i risultati dopo ogni test, oppure eseguire diversi test e generare un rapporto finale in una pagina formattata come una tabella \ac{HTML}. Per visualizzare un report si utilizzi la funzione \var{eco} oppure si utilizzi la funzione \var{run()}:

\begin{code}
echo $this->unit->run($test, $expected_result);
\end{code}

Per ottenere un rapporto di tutti i test:

\begin{code}
echo $this->unit->report();
\end{code}

Se invece di una formattazione \ac{HTML} si preferisce ottenere il report come raw data (dati grezzi), si può utilizzare un array:

\begin{code}
echo $this->unit->result();
\end{code}

\section*{Modalit\'a Strict}

La classe Unit Test come impostazione predefinita effettua un confronto letterale di tipo loose. Si consideri questo esempio:

\begin{code}
$this->unit->run(1, TRUE);
\end{code}

Il test valuta un valore intero (1) ma il risultato atteso è un booleano (TRUE). Il linguaggio PHP, però, grazie alla sua tipizzazione dei dati loose valuterà il codice di cui sopra come TRUE utilizzando un test di uguaglianza normale:

\begin{code}
if (1 == TRUE) echo 'This evaluates as true';
\end{code}

Se si preferisce, si può impostare la classe sulla modalità Strict per effettuare un confronto sia con il tipo di dati, che il suo valore

\begin{code}
$this->unit->use_strict(TRUE);
\end{code}

Ora, con questa modalità, il test di prima darà esito negativo:

\begin{code}
if (1 === TRUE) echo 'This evaluates as FALSE';
\end{code}

\section*{Unit Testing}

\'E possibile disabilitare alcuni test nei propri script utilizzando:

\begin{code}
$this->unit->active(FALSE)
\end{code}

\section*{Visualizzare Unit Testing}

Quando vengono visualizzati i risultati dei test, i seguenti elementi saranno messi in evidenza per impostazione predefinita:

\begin{itemize}
\item \verb|test_name|: il nome del test
\item \verb|test_datatype|: il tipo di dati del test
\item \verb|res_datatype|: il tipo di dati atteso
\item result: il risultato atteso
\item file: il nome del file
\item line: il numero di linea
\item notes: eventuale descrizione
\end{itemize}

Si può modificare uno o più elementi sopraccitati con la funzione \verb|$this->unit->set_items()}|. Per esempio se si desidera visualizzare solo il nome del test e il risultato:

\begin{code}
$this->unit->set_test_items(array('test_name', 'result'));
\end{code}

\section*{Creare un template}

Se si vuole modificare la formattazione con cui vengono visualizzati i risultati del test si può impostare un proprio template. Qui di seguito si mostra un semplice template in cui sono utilizzate delle pseudo variabili:

\begin{html}
$str = '
<table border="0" cellpadding="4" cellspacing="1">
    {rows}
        <tr>
        <td>{item}</td>
        <td>{result}</td>
        </tr>
    {/rows}
</table>';

$this->unit->set_template($str);
\end{html}

Nota: il proprio template deve essere dichiarato prima di eseguire il processo Unit Test.