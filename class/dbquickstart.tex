%inizio Database Quick Start: Example Code
\section*{Introduzione}
Qui di seguito verranno descritti alcuni esempi su come viene utilizzata la classe Database. Per informazioni dettagliate si rimanda alla lettura delle singole sezioni che descrivono ogni funzione.

\section*{Query Standard con risultati multipli (object)}
Verrà mostrato un codice di una query standard con risultati multipli (versione Object). La funzione \var{resul()} restituisce un array di oggetti. Esempio \var{\$row->title}.

\begin{code}
$query = $this->db->query('SELECT name, title, email FROM my_table');

foreach ($query->result() as $row)
{
    echo $row->title;
    echo $row->name;
    echo $row->email;
}

echo 'Totale dei risultati: ' . $query->num_rows();
\end{code}

\section*{Query Standard con risultati multipli (array)}
La query standard con risultati multipli, ma che utilizza l'array per i propri dati, prevede una sintassi leggermente differente. In questo caso la funzione \verb|result_array()| restituisce un array di indici di array, come per esempio \var{\$row['title']}:

\begin{code}
$query = $this->db->query('SELECT name, title, email FROM my_table');

foreach ($query->result_array() as $row)
{
    echo $row['title'];
    echo $row['name'];
    echo $row['email'];
}
\end{code}

Quando si esegue una query solitamente non vengono prodotti risultati su schermo: questo rende difficile capire se il comportamento della query sia corretto. La funzione \verb|num_rows()| aiuta in tal senso perché fornisce il numero di righe prodotto dalla query, ovvero i risultati ottenuti.

\begin{code}
$query = $this->db->query("LA TUA QUERY");

if ($query->num_rows() > 0)
{
   foreach ($query->result() as $row)
   {
      echo $row->title;
      echo $row->name;
      echo $row->body;
   }
}
\end{code}

La query standard con un singolo risultato presenta la funzione \var{row()} che resistuisce un oggetto:

\begin{code}
$query = $this->db->query('SELECT name FROM my_table LIMIT 1');

$row = $query->row();
echo $row->name;
\end{code}

La query standard con un singolo risultato può fare uso anche di un array. In questo caso la funzione \verb|row_array()| restituisce un array:

\begin{code}
$query = $this->db->query('SELECT name FROM my_table LIMIT 1');

$row = $query->row_array();
echo $row['name'];
\end{code}

\section*{Inserimento dei dati in modalit\'a standard}
\begin{code}
$sql = "INSERT INTO mytable (title, name) 
        VALUES (".$this->db->escape($title).", ".$this->db->escape($name).")";

$this->db->query($sql);

echo $this->db->affected_rows();
\end{code}

\section*{Query Active Record}
Il pattern Active Record fornisce un sistema per semplificare le interrogazioni al database. La funzione \var{get()} per esempio restituisce tutti i risultati disponibili in una tabella. La classe Active Record contiene un nutrita serie di funzioni per lavorare agevolmente con un database.

\begin{code}
$query = $this->db->get('table_name');

foreach ($query->result() as $row)
{
    echo $row->title;
}
\end{code}

\section*{Inserimento dei dati con Active Record}
\begin{code}
$data = array(
               'title' => $title,
               'name' => $name,
               'date' => $date
            );

$this->db->insert('mytable', $data); 

// Produce: INSERT INTO mytable (title, name, date) VALUES ('{$title}', '{$name}', '{$date}')
\end{code}
%fine Database Quick Start: Example Code