%inizio Table data
\section*{Tabella metadati}
Per ottenere informazioni sulle proprie tabelle, CodeIgniter fornisce due preziose funzioni:

\begin{itemize}
\item \verb|$this->db->list_tables()| restituisce un array che contiene i nomi di tutte le tabelle nel database a cui si è connessi. Per esempio:

\begin{code}
$tables = $this->db->list_tables();

foreach ($tables as $table)
{
   echo $table;
}
\end{code}

\item \verb|$this->db->table_exists()| alcune volte è di aiuto conoscere se una determinata tabella esiste prima di eseguire un'operazione su di essa. Questa funzione restituisce un valore booleano TRUE/FALSE:

\begin{code}
if ($this->db->table_exists('table_name'))
{
   // altro codice...
}
\end{code}

Si sostituisca il parametro \verb|table_name| con il nome della tabella di cui si vuole verificare l'esistenza.
\end{itemize}

%fine Table Data

%inizio Field Data
\section*{Campi metadati}

\begin{itemize}
\item \verb|$this->db->list_fields()| restituisce un array che contiene i nomi dei campi. Questa query può essere chiamata in due modi:

\begin{enumerate}
\item \'E possibile fornire il nome della tabella e richiamarla \verb|$this->db->|

\begin{code}
$fields = $this->db->list_fields('table_name');

foreach ($fields as $field)
{
   echo $field;
}
\end{code}

\item I nomi dei campi associati ad una qualsiasi query che si sta eseguendo, possono essere raccolti chiamando la funzione dall'oggetto risultato della query:

\begin{code}
$query = $this->db->query('SELECT * FROM some_table'); 

foreach ($query->list_fields() as $field)
{
   echo $field;
}
\end{code}
\end{enumerate}

\item \verb|$this->db->field_exists()| alcune volte è di aiuto conoscere se un particolare campo esiste prima di eseguire un'azione: questa funzione restituisce un valore booleano TRUE/FALSE. Si sostituisca il campo \verb|field_name| con il nome dell'attributo che si desidera verificare.

\begin{code}
if ($this->db->field_exists('field_name', 'table_name'))
{
   // altro codice...
}
\end{code}

\item \verb|$this->db->field_data()| restituisce un array di oggetti che contiene informazioni sul campo selezionato come per esempio, il tipo di attributo, la lunghezza massima del campo, ecc. Questa funzione può non essere disponibile in tutti i database:

\begin{code}
$fields = $this->db->field_data('table_name');

foreach ($fields as $field)
{
   echo $field->name;
   echo $field->type;
   echo $field->max_length;
   echo $field->primary_key;
}
\end{code}

Se si è già eseguita una query, è possibile utilizzare un oggetto (risultato della query) invece di fornire il nome della tabella.

\begin{code}
$query = $this->db->query("LA TUA QUERY");
$fields = $query->field_data();
\end{code}

I seguenti metadati sono disponibili nella funzione se supportati dal proprio database:

\begin{enumerate}
\item \verb|name|: nome della colonna (attributo)
\item \verb|max_length|: massima lunghezza della colonna
\item \verb|primary_key|: fornisce il valore ``1'' se l'attributo è una chiave primaria (primary key)
\item \verb|type|: il tipo di attributo
\end{enumerate}
\end{itemize}
%fine Field Data