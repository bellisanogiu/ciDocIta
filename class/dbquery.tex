%inizio Query
\section*{Query}
In questa sezione verranno trattare le query ovvero quelle istruzioni che permettono di ``interrogare'' i database per ottenere/modificare una grande mole di informazioni.

\verb|$this->db->query()| è l'istruzione basilare per eseguire una query. La sintassi dell'istruzione è:

\begin{code}
$this->db->query('INSERISCI QUI LA TUA QUERY');
\end{code}

La funzione \var{query()} restituisce un oggetto quando vengono eseguite le query di tipo read, utilizzate per visualizzare i risultati (per esempio \var{SELECT}). Quando invece vengono eseguite le query di tipo write, la funzione semplicemente restituisce il valore booleano TRUE oppure FALSE, a seconda del successo o meno dell'operazione. Quando si recuperano i dati, tipicamente si può assegnare la query alla propria variabile come nel seguente:

\begin{code}
$query = $this->db->query('INSERISCI QUI LA TUA QUERY');
\end{code}

\verb|$this->db->simple_query()| è la versione semplificata della funzione \verb|$this->db->query()|. Essa restituisce solo i valori TRUE oppure FALSE a seconda del successo o meno della query. Questa funzione perciò non restituisce una lista di risultati, né imposta le query timer, compila i dati bind, o memorizza le proprie query per il debug: viene semplicemente utilizzata per inviare/sottoporre una query. Questa funzione viene implementata raramente.

\section*{Prefisso del database}
Se si è configurato manualmente un prefisso per il database e lo si vuole anteporre ad un nome di tabella per utilizzarlo per esempio in una query nativa SQL, è possibile utilizzare il seguente comando:

\begin{code}
$this->db->dbprefix('tablename');
// restituisce prefisso_tablename
\end{code}

Se per qualsiasi ragione è necessario cambiare il prefisso, senza creare una nuova connessione, si può adoperare la funzione:

\begin{code}
$this->db->set_dbprefix('newprefix');

$this->db->dbprefix('tablename');
// restituisce nuovoprefisso_tablename
\end{code}

\section*{Proteggere gli identificatori}
In molti database, è consigliabile proteggere la tabella e i record, per esempio con i backtick in MySQL (si veda la definizione\vpageref{def:Backtick}). Con Active Record, le query sono automaticamente protette, tuttavia se si ha bisogno di proteggere manualmente un identificatore è possibile utilizzare:

\begin{code}
$this->db->protect_identifiers('table_name');
\end{code}

\begin{deftab}{Backtick}{L'operatore di esecuzione backtick è costituito dagli apici inversi (``). Si noti che non si tratta di doppi apici o apostrofi. L'istruzione contenuta all'interno dei backtick verrà eseguito come se fosse un comando di shell: il risultato dovrà essere assegnato ad una variabile e visualizzato eventualmente con il comando echo. L'uso dell'operatore backtick è identico alla funzione shell\_exec() e diversamente da altri linguaggi non possono essere utilizzati all'interno di stringhe delimitate da doppi apici. Per finire, l'operatore backtick viene disabilitato quando la modalità sicura è attiva oppure se shell\_exec() è disabilitata.}
\end{deftab}
\normalsize

Questa funzione si può anche aggiungere al prefisso della propria tabella, ipotizzando che un prefisso sia indicato nel proprio file di configurazione. Per abilitare l'uso del prefisso è necessario impostare su TRUE il secondo parametro:
 
\begin{code}
$this->db->protect_identifiers('table_name', TRUE);
\end{code}

\section*{Escape delle query}
\'E una buona pratica di sicurezza effettuare l'escape dei dati prima di inviarli al proprio database. CodeIgniter ha tre metodi utili a tale scopo.

\begin{itemize}
\item \verb|$this->db->escape()| questa funzione determina il tipo di dati in modo che si possa fare l'escape solo dei dati stringa. Essa aggiunge le singole virgolette attorno al dato:

\begin{code}
$sql = "INSERT INTO table (title) VALUES(".$this->db->escape($title).")";
\end{code}

\item \verb|$this->db->escape_str()| questa funzione effettua l'escape dei dati che gli vengono passati indipendentemente dal tipo:

\begin{code}
$sql = "INSERT INTO table (title) VALUES('".$this->db->escape_str($title)."')";
\end{code}

\item \verb|$this->db->escape_like_str()| questa funzione deve essere usata quando le stringhe contengono la condizione LIKE e i wildcard (\verb| %, _|).

\begin{code}
$search = '20% raise';
$sql = "SELECT id FROM table WHERE column LIKE '%".$this->db->escape_like_str($search)."%'";
\end{code}
\end{itemize}

\section*{Query binding}
I Binding consentono di semplificare la sintassi di query lasciando che il sistema inserisca le query per noi. Si consideri il seguente esempio:

\begin{code}
$sql = "SELECT * FROM some_table WHERE id = ? AND status = ? AND author = ?"; 

$this->db->query($sql, array(3, 'live', 'Rick'));
\end{code}

I ``punto di domanda'' nella query vengono sostituiti automaticamente con i valori presenti nel secondo parametro dell'array della funzione \var|query()|. Il secondo vantaggio nell'utilizzare Bind è che i valori sono automaticamente controllati, permettendo così query più sicure. In pratica, non ci si deve più preoccupare di effettuare l'escape manualmente perché sarà l'engine di CodeIgniter a farlo automaticamente.
%fine {Query}