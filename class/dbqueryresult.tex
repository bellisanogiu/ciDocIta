%inizio Generating Query Results
\section*{I risultati delle query}
Ci sono diverse strade per generare i risultati delle query: qui analizzeremo tutte le funzioni messe a disposizione da CodeIgniter:

\begin{itemize}
\item \verb|result()| questa funzione restituisce il risultato di una query come un array di oggetti o un array vuoto in caso di fallimento; solitamente viene utilizzata in un loop foreach come il seguente:

\begin{code}
$query = $this->db->query("LA TUA QUERY");

foreach ($query->result() as $row)
{
   echo $row->title;
   echo $row->name;
   echo $row->body;
}
\end{code}

La funzione è un alias di \verb|result_object()|. In tutti i casi se si esegue una query che potrebbe non produrre il risultato, è consigliabile verificare prima il risultato:

\begin{code}
$query = $this->db->query("LA TUA QUERY");

if ($query->num_rows() > 0)
{
   foreach ($query->result() as $row)
   {
      echo $row->title;
      echo $row->name;
      echo $row->body;
   }
}
\end{code}

\'E anche possibile passare una stringa alla funzione \verb|result()| che rappresenta una classe per istanziare ogni risultato oggetto (nota: questa classe deve essere caricata):

\begin{code}
$query = $this->db->query("SELECT * FROM users;");

foreach ($query->result('User') as $row)
{
   echo $row->name; // chiama gli attributi
   echo $row->reverse_name(); // o i metodi definiti nella classe User
}
\end{code}

\item \verb|result_array()| questa funzione restituisce il risultato della query come un array o un array vuoto quando non viene prodotto alcun risultato. Solitamente questa funzione viene utilizzata in un loop foreach come:

\begin{code}
$query = $this->db->query("LA TUA QUERY");

foreach ($query->result_array() as $row)
{
   echo $row['title'];
   echo $row['name'];
   echo $row['body'];
}
\end{code}

\item \verb|row()| restituisce come risultato una singola riga. Se la query ha più di una riga, sarà processata solo la prima. Il risultato è restituito con un oggetto. Un esempio:

\begin{code}
$query = $this->db->query("LA TUA QUERY");

if ($query->num_rows() > 0)
{
   $row = $query->row(); 

   echo $row->title;
   echo $row->name;
   echo $row->body;
}
\end{code}

Se si desidera che venga restituita una specifica riga tra i vari risultati, è possibile indicarla nel primo parametro sotto forma di numero intero:

\begin{code}
$row = $query->row(5);
\end{code}

\'E anche possibile aggiungere un secondo parametro con una stringa, ovvero il nome della classe per istanziare la riga:

\begin{code}
$query = $this->db->query("SELECT * FROM users LIMIT 1;");

$query->row(0, 'User')
echo $row->name; // chiama gli attributi
echo $row->reverse_name(); // o i metodi definiti nella classe User
\end{code}

\item \verb|row_array()| è una funzione simile a \verb|row()| eccetto per il fatto che restituisce un array. Per esempio:

\begin{code}
$query = $this->db->query("LA TUA QUERY");

if ($query->num_rows() > 0)
{
   $row = $query->row_array(); 

   echo $row['title'];
   echo $row['name'];
   echo $row['body'];
}
\end{code}

Se si desidera che venga restituita una specifica riga, è possibile specificare il numero della riga nel primo parametro:

\begin{code}
$row = $query->row_array(5);
\end{code}

In aggiunta ci si può spostare tra i risultati in avanti/indietro/all'inizio/alla fine utilizzando questi metodi:

\begin{code}
$row = $query->first_row()
$row = $query->last_row()
$row = $query->next_row()
$row = $query->previous_row()
\end{code}

Per impostazione predefinita questi metodi restituiscono un oggetto a meno di mettere la parola ``array'' nel parametro:

\begin{code}
$row = $query->first_row('array')
$row = $query->last_row('array')
$row = $query->next_row('array')
$row = $query->previous_row('array')
\end{code}
\end{itemize}

\section*{Funzioni Helper: i risultati}
\begin{itemize}
\item \verb|$query->num_rows()| fornisce il numero di righe restituite dalla query. In questo esempio \verb|$query| è la variabile a cui viene assegnato il risultato dell'interrogazione:


\begin{code}
$query = $this->db->query('SELECT * FROM my_table');

echo $query->num_rows();
\end{code}

\item \verb|$query->num_fields()| viene restituito il numero di colonne (attributi) della query. \'E necessario essere sicuri di chiamare la funzione utilizzando il risultato della query memorizzato in \var{\$query}:

\begin{code}

$query = $this->db->query('SELECT * FROM my_table');

echo $query->num_fields();
\end{code}

\item \verb|$query->free_result()| questa funzione libera la memoria associata al risultato e cancella il risultato della risorsa individuata dall'ID. Normalmente \ac{PHP} libera automaticamente la memoria al termine dell'esecuzione dello script, tuttavia, se si eseguono numerose interrogazioni in un particolare script è consigliabile eliminare subito questi risultati. Per esempio:

\begin{code}
$query = $this->db->query('SELECT title FROM my_table');

foreach ($query->result() as $row)
{
   echo $row->title;
}
$query->free_result(); // L'oggetto $query non è più disponibile

$query2 = $this->db->query('SELECT name FROM some_table');

$row = $query2->row();
echo $row->name;
$query2->free_result(); // L'oggetto $query2 non è ora disponibile
\end{code}
\end{itemize}
%fine Generating Query Results