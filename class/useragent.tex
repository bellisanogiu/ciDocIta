%************************************************
\section{Classe User Agent}
\label{class:useragent}
%************************************************

La classe User Agent fornisce degli utili strumenti per identificare le informazioni riguardanti i browser, i device mobile e i robot che si collegano al proprio sito. In aggiunta a questo, la classe rivela altri dati come il linguaggio e la codifica dei caratteri supportata.

La classe viene inizializzata nel proprio controller con la funzione \verb|$this->load->library('user_agent')|. Una volta caricata, l'oggetto sarà disponibile con la funzione \var{\$this->agent}.

\section*{Definizione}

Le definizioni riguardanti lo ``user agent'' sono memorizzate nel file \verb|/application/config/user_agents.php| in cui è possibile aggiungere ulteriori elementi se necessario.

Per esempio, quando la classe è inizializzata è possibile determinare se lo user agent che naviga nel sito è un browser web, un dispositivo mobile o un robot. \'E anche possibile ottenere altre informazioni sulla piattaforma come il sistema operativo utilizzato.

\begin{code}
$this->load->library('user_agent');

if ($this->agent->is_browser())
{
    $agent = $this->agent->browser().' '.$this->agent->version();
}
elseif ($this->agent->is_robot())
{
    $agent = $this->agent->robot();
}
elseif ($this->agent->is_mobile())
{
    $agent = $this->agent->mobile();
}
else
{
    $agent = 'Unidentified User Agent';
}

echo $agent;

echo $this->agent->platform(); // Platform info (Windows, Linux, Mac, etc.)
\end{code}

\section*{Reference delle funzioni}

\begin{itemize}
\item \verb|$this->agent->is_browser()| restituisce un valore booleano TRUE/FALSE se il browser web è un user agent conosciuto o meno:

\begin{code}
if ($this->agent->is_browser('Safari'))
{
    echo 'You are using Safari.';
}
else if ($this->agent->is_browser())
{
    echo 'You are using a browser.';
}
\end{code}

La stringa ``Safari'', presente in questo esempio è una chiave dell'array che contiene la lista deii browser web. Il file \verb|application/config/user_agents.php| contiene la lista di tutti i browser a cui è possibile aggiungerne di nuovi o modificare quelli esistenti.

\item \verb|$this->agent->is_mobile()| restituisce un valore booleano TRUE/FALSE se il dispositivo mobile è uno user agent conosciuto o meno:

\begin{code}
if ($this->agent->is_mobile('iphone'))
{
    $this->load->view('iphone/home');
}
else if ($this->agent->is_mobile())
{
    $this->load->view('mobile/home');
}
else
{
    $this->load->view('web/home');
}
\end{code}

\item \verb|$this->agent->is_mobile()| restituisce un valore booleano TRUE/FALSE se il dispositivo mobile è uno user agent conosciuto o meno:

\begin{code}
if ($this->agent->is_mobile('iphone'))
{
    $this->load->view('iphone/home');
}
else if ($this->agent->is_mobile())
{
    $this->load->view('mobile/home');
}
else
{
    $this->load->view('web/home');
}
\end{code}

\item \verb|$this->agent->is_robot()| restituisce un valore booleano TRUE/FALSE se il robot è uno user agent conosciuto o meno:

Nota: La libreria degli user agent contiene solo le definizioni dei robot più comuni. Non è certo un elenco esaustivo dei bot: ne esistono centinaia e verificare ognuno di questi non sarebbe molto efficiente. In tutti i casi, se si scopre che alcuni bot che normalmente visitano il sito mancano nella lista è possibile aggiungerli manualmente al file \verb|application/config/user_agents.php|.

\item \verb|$this->agent->is_referral()| restituisce il valore booleano TRUE/FALSE se lo user agent è riferito a quello di un altro sito.

\item \verb|$this->agent->browser()| restituisce una stringa che contiene il nome del browser web che visita il proprio sito.

\item \verb|$this->agent->version()| restituisce una stringa che contiene il numero di versione del browser web che visita il proprio sito.

\item \verb|$this->agent->mobile()| restituisce una stringa che contiene il nome del dispositivo mobile che visita il proprio sito.

\item \verb|$this->agent->robot()| restituisce una stringa che contiene il nome del robot che visita il proprio sito.

\item \verb|$this->agent->platform()| restituisce una stringa che contiene il nome sistema operativo (Linux, Windows, OS X, etc.) utilizzato per visitare il proprio sito.

\item \verb|$this->agent->referrer()| è una stringa che viene prodotta quando lo user agent è dovuto alla visita di un altro sito:

\begin{code}
if ($this->agent->is_referral())
{
    echo $this->agent->referrer();
}
\end{code}

\item \verb|$this->agent->agent_string()| restituisce una stringa che contiene la stringa completa user agent. Un esempio:

\begin{code}
Mozilla/5.0 (Macintosh; U; Intel Mac OS X; en-US; rv:1.8.0.4) Gecko/20060613 Camino/1.0.2
\end{code}

\item \verb|$this->agent->accept_lang()| determina se lo user agent accetta una particolare lingua. Questa funzione non è in genere molto affidabile in quanto alcuni browser non forniscono informazioni sulla lingua, e tra quelli che lo fanno, il dato non è sempre preciso.

\begin{code}
if ($this->agent->accept_lang('en'))
{
    echo 'You accept English!';
}
\end{code}

\item \verb|$this->agent->accept_charset()| determina se lo user agent accetta una codifica dei caratteri. Questa funzione non è in genere molto affidabile in quanto alcuni browser non forniscono informazioni sulla codifica dei caratteri, e tra quelli che lo fanno, il dato non è sempre preciso. Per esempio:

\begin{code}
if ($this->agent->accept_charset('utf-8'))
{
    echo 'You browser supports UTF-8!';
}
\end{code}
\end{itemize}