%************************************************
\section{Classe Benchmarking}
\label{class:benchmark}
%************************************************

La classe Benchmarking svolge un semplice quanto prezioso compito: calcola la differenza di tempo tra due marcatori fissati. Il valore, espresso in secondi, permette di effettuare delle stime sulla velocità del progetto sviluppato, calcolando il tempo necessario per processare una serie di istruzioni, o caricare una vista ricca di dati e immagini. Non meno importante è la funzione che misura la quantità di memoria impegnata dal progetto, che dovrà essere sempre opportunamente dimensionato all'hardware del server su cui viene eseguito. Questa classe non ha bisogno, a differenza di altre, di essere abilitata manualmente, ma al contrario risulta sempre attiva: essa viene avviata non appena si utilizza il framework e termina poco prima che la classe output invii i dati alla Vista che li mostra sul browser.

\subsection*{Utilizzare la classe}
La classe Benchmarking può essere utilizzata con i Controller, Viste e Modelli seguendo i seguenti passi:

\begin{enumerate}
\item creare un punto marcatore iniziale
\item creare un punto marcatore finale
\item eseguire la funzione ``elapsed time'' per visualizzare i risultati
\end{enumerate}

Un esempio del codice vero e proprio è il seguente:

\begin{code}
$this->benchmark->mark('code_start');

// spazio per il codice

$this->benchmark->mark('code_end');

echo $this->benchmark->elapsed_time('code_start', 'code_end');
\end{code}

Nota: le parole \verb|code_start| e \verb|code_end| sono arbitrarie: è possibile utilizzare qualsiasi parola, che indichi preferibilmente in maniera chiara l'inizio e la fine di un blocco di codice da analizzare. Nell'esempio di qui sotto sono state utilizzate tre parole chiave che non risultano intuitive nel definire i marcatori di inizio e fine codice. Comunque qualsiasi scelta si faccia, CodeIgniter calcolerà il tempo correttamente:

\begin{code}
$this->benchmark->mark('cane');

// spazio per il codice

$this->benchmark->mark('gatto');

// spazio per il codice

$this->benchmark->mark('uccello');

echo $this->benchmark->elapsed_time('cane', 'gatto');
echo $this->benchmark->elapsed_time('gatto', 'uccello');
echo $this->benchmark->elapsed_time('cane', 'uccello');
\end{code}

\subsection*{Profilazione dei punti benchmark}
Un aspetto interessante di questa classe è che si possono utilizzare i dati del benchmark condividendoli con il Profiler/vref{cap:profilazione}. \'E sufficiente collegare in coppia i propri marcatori, facendo terminare i rispettivi nomi con le stringhe \verb|_start| e \verb|_end|. Altro aspetto importante è che ogni coppia di marcatori deve avere necessariamente lo stesso nome. Per esempio:

\begin{code}
$this->benchmark->mark('mio_marcatore_start');

// spazio per il codice

$this->benchmark->mark('mio_marcatore_end'); 

$this->benchmark->mark('altro_marcatore_start');

// spazio per il codice

$this->benchmark->mark('altro_marcatore_end');
\end{code}

Maggiori informazioni sono disponibili nel capitolo dedicato alla Profilazione\vpageref{cap:profilazione}.

\subsection*{Il tempo totale di esecuzione}
Se si desidera visualizzare il tempo trascorso dal momento in cui CodeIgniter viene eseguito per la prima volta sino al momento dell'output finale verso il browser, è sufficiente porre in una delle proprie Viste la seguente istruzione:

\begin{code}
<?php echo $this->benchmark->elapsed_time();?>
\end{code}

Si noterà che la funzione è la stessa, utilizzata negli esempi precedenti per calcolare il tempo tra due punti, ma in questo caso, non si specifica alcun argomento. Quando i parametri sono assenti, CodeIgniter non termina il benchmark fino a quando il risultato finale non viene inviato al browser. Non importa dove si utilizzerà la chiamata di funzione, il timer continuerà a funzionare fino alla fine. Un modo alternativo per mostrare il tempo trascorso nei file di vista è quello di utilizzare questa pseudo-variabile, che evita di ricorrere al \ac{PHP} puro:

\begin{code}
{elapsed_time}
\end{code}

\section*{Il consumo di memoria}
Il consumo di memoria invece è un altro aspetto importante che è opportuno monitorare nei progetti che coinvolgono ingenti quantità di dati. Se l'installazione del \ac{PHP} sul sistema è configurata con l'opzione \verb|--enable-memory-limit|, sarà possibile visualizzare il consumo di memoria dell'intero sistema, inserendo nelle Viste desiderate:

\begin{code}
<?php echo $this->benchmark->memory_usage();?>
\end{code}

Nota: questa funzione può essere utilizzata solo nelle Viste, dove verrà visualizzata la quantità di memoria consumata dal progetto. Un metodo alternativo per ottenere lo stesso risultato, prevedo l'uso del pseudo-codice:

\begin{code}
{memory_usage}
\end{code}