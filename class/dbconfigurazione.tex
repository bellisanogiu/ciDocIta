%inizio Database Configuration

\section*{Configurazione del database}

La configurazione del database avviene inserendo i dati nel file \fil{/application/config/database.php} come per esempio lo username e la password per stabilire una connessione con la propria base di dati. Il prototipo del file di configurazione è il seguente:

\begin{code}
$db['default']['hostname'] = "localhost";
$db['default']['username'] = "root";
$db['default']['password'] = "";
$db['default']['database'] = "database_name";
$db['default']['dbdriver'] = "mysql";
$db['default']['dbprefix'] = "";
$db['default']['pconnect'] = TRUE;
$db['default']['db_debug'] = FALSE;
$db['default']['cache_on'] = FALSE;
$db['default']['cachedir'] = "";
$db['default']['char_set'] = "utf8";
$db['default']['dbcollat'] = "utf8_general_ci";
$db['default']['swap_pre'] = "";
$db['default']['autoinit'] = TRUE;
$db['default']['stricton'] = FALSE;
\end{code}

Il file di configurazione si basa su un array multi-dimensionale piuttosto che su uno più semplice. Il motivo di questa scelta è quello di permettere la memorizzare di più insiemi di valori di connessione. Se, per esempio, si eseguono più ambienti (sviluppo, produzione, test, ecc) con un'unica installazione del framework, è possibile impostare un gruppo di connessione per ciascuno di essi, e in seguito passare tra i gruppi in base alle necessità. Ad esempio, per impostare un ambiente di ``test'' si procede nel seguente modo:

\begin{code}
$db['test']['hostname'] = "localhost";
$db['test']['username'] = "root";
$db['test']['password'] = "";
$db['test']['database'] = "database_name";
$db['test']['dbdriver'] = "mysql";
$db['test']['dbprefix'] = "";
$db['test']['pconnect'] = TRUE;
$db['test']['db_debug'] = FALSE;
$db['test']['cache_on'] = FALSE;
$db['test']['cachedir'] = "";
$db['test']['char_set'] = "utf8";
$db['test']['dbcollat'] = "utf8_general_ci";
$db['test']['swap_pre'] = "";
$db['test']['autoinit'] = TRUE;
$db['test']['stricton'] = FALSE;
\end{code}

Nota: l'array multidimensionale, ha come valore ``test''. 

Quindi per utilizzare questa impostazione globalmente, si modifica il parametro seguente sempre nel file config:

\begin{code}
$active_group = "test";
\end{code}

Il nome ``test'' è arbitrario: quindi è possibile impostare qualsiasi nome. Per impostazione predefinita, si è usata la parola ``default'' per la connessione primaria, ma anche questo può essere rinominato in qualcosa di più rilevante per il proprio progetto.

\section*{Active Record}
La classe Active Record si può abilitare globalmente o meno agendo sul parametro di configurazione \verb|$active_record| nel file di configurazione del database. Se non si utilizza la classe Active Record, si imposti il parametro sul valore FALSE: questo permetterà al sistema di consumare meno risorse quando le classi database saranno inizializzate. Se invece si vuole sfruttare questa peculiarità di CodeIgniter sarà sufficiente impostare il valore su TRUE:

\begin{code}
$active_record = TRUE;
\end{code}

Nota: Alcune classi di CodeIgniter come per esempio Sessions, richiedono che Active Recortd sia abilitato per accedere ad alcune sue funzionalità.

\section*{Lista dei valori}
\begin{itemize}
\item hostname. L'indirizzo del server in cui si trova il database, spesso identificato con localhost o 127.0.0.1
\item username. Il nome (login) con cui effettuare l'accesso
\item password. La parola segreta per accedere al database
\item database. Il nome del database a cui ci si vuole connettere
\item dbdriver. La tipologia di database utilizzato: mysql, postgres, odbc, ecc. Il nome deve essere specificato scrivendolo in minuscolo
\item dbprefix. Indica un prefisso opzionale da aggiungere al nome della tabella quando si eseguono delle query Active Recors. Questo permette a diverse installazioni di CodeIgniter di condividere lo stesso database
\item pcconnect. Viene utilizzato per indicare delle connessioni persistenti o meno: valore booleano TRUE/FALSE
\item \verb|db_debug|. Viene utilizzato per indicare per visualizzare o meno gli errori del database: valore booleano TRUE/FALSE
\item \verb|cache_on|. Viene utilizzato per abilitare o meno la cache delle query. Si veda anche la classe Database Caching
\item cachedir. Il percorso assoluto della directory della query cache database
\item \verb|char_set|. Il set di caratteri utilizzato con il database
\item dbcollat. Le regole di confronto tra caratteri utilizzate nella comunicazione con il database. Questo parametro per MySQL e MySQLi è usato solo come backup se il proprio server funziona con una versione del PHP inferiore alla v.5.2.3. oppure se MySQL è inferiore alla v.5.0.7 (e le query di creazione della tabella sono realizzate con DB Forge). Esiste una incompatibilità nel PHP con la funzione \verb|mysql_real_escape_string()| che rende il proprio sito vulnerabile alle infezioni di tipo SQL se si usa un set di caratteri multi-byte e se si usa una versione inferiore di MySQL e PHP. I siti che utilizzano come set di caratteri per il database come Latin-1 o UTF-8 e collation non sono vulnerabili a questa minaccia
\item \verb|swap_pre|. Il prefisso di default della tabella che può essere scambiato con \var{dbprefix}. Questo parametro è utile per le applicazioni distribuite in cui le query devono essere scritte manualmente e devono avere un prefisso personalizzabile dall'utente finale
\item autoinit. Con questo parametro si stabilisce se si deve impostare una connessione automatica al database quando viene caricata la libreria. Se è viene utilizzato il valore FALSE la connessione avverrà prima di eseguire la prima query
\item stricton. Stabilisce se forzare la connessione ``Strict Mode'', consigliato per garantire il rigoroso rispetto di SQL durante lo sviluppo di un'applicazione
\item port. Il numero di porta della base di dati: si utilizza aggiungendo una linea nel file di configurazione relativa all'array del database.

\begin{code}
$db['default']['port'] = 5432;
\end{code}
\end{itemize}

A seconda di quale piattaforma di database si utilizza (MySQL, Postgres, ecc) non saranno necessari tutti i valori precedentemente indicati. Ad esempio, quando si utilizza SQLite non si dovrà fornire un nome utente o la password, e il nome del database sarà il percorso del file di database. Le informazioni di cui sopra si riferiscono ad un sistema basato su MySQL.
%fine Database Configuration