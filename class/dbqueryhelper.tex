%inizio Query Helper Functions
\section*{Funzioni Helper Query}

\begin{itemize}
\item \verb|$this->db->insert_id()| il numero ID di inserimento quando si esegue un record viene aggiunto nel database.

\item \verb|$this->db->affected_rows()| visualizza il numero di righe interessate quando vengono processate le query di tipo  write (INSERT, UPDATE, ecc.).

Nota: in MySQL "DELETE FROM TABLE" restituisce 0 (zero) righe. La classe di database ha un piccolo hack che permette di restituire il giusto numero di righe interessate. Per impostazione predefinita, questo hack è abilitato, ma può essere disattivato nel file driver di database

\item \verb|$this->db->count_all()| permette di determinare il numero di righe in una specifica tabella. Si deve indicare il nome della tabella come primo parametro:

\begin{code}
echo $this->db->count_all('my_table');

// Produce un intero come 25
\end{code}

\item \verb|$this->db->platform()| visualizza il tipo di database utilizzato (MySQL, MS SQL, Postgres, etc).

\begin{code}
echo $this->db->platform();
\end{code}

\item \verb|$this->db->version()| visualizza la versione del database utilizzato:

\begin{code}
$this->db->version()
\end{code}

\item \verb|$this->db->last_query()| restituisce l'ultima query eseguita (la stringa query, non il risultato):

\begin{code}
$str = $this->db->last_query();

// Produce: SELECT * FROM qualche tabella...
\end{code}

\item \verb|$this->db->insert_string()| questa funzione semplifica il processo di scrittura degli inserimenti nel database. Viene restituita una stringa di inserimento nel formato SQL.

\begin{code}
$data = array('name' => $name, 'email' => $email, 'url' => $url);

$str = $this->db->insert_string('table_name', $data);
\end{code}

Il primo parametro è il nome della tabella, il secondo è un array associativo che contiene i dati da inserire. L'esempio di qui sopra produce:

\begin{code}
INSERT INTO table_name (name, email, url) VALUES ('Rick', 'rick@example.com', 'example.com')
\end{code}

Nota: i valori subiscono automaticamente l'escape, aspetto che rende le query sicure.

\item \verb|$this->db->update_string()| questa funzione semplifica l'aggiornamento del database. Viene restituita un stringa di aggiornamento formattata SQL. Per esempio:

\begin{code}
$data = array('name' => $name, 'email' => $email, 'url' => $url);

$where = "author_id = 1 AND status = 'active'"; 

$str = $this->db->update_string('table_name', $data, $where);
\end{code}

Il primo parametro è il nome della tabella, mentre il secondo è un array associativo con i dati da aggiornare. Il terzo parametro è la clausola WHERE. L'esempio di qui sopra produce:

\begin{code}
UPDATE table_name SET name = 'Rick', email = 'rick@example.com', url = 'example.com' WHERE author_id = 1 AND status = 'active'
\end{code}

Nota: i valori subiscono automaticamente l'escape, aspetto che rende le query sicure.
\end{itemize}
%fine Query Helper Functions