%************************************************
\chapter{URL di CodeIgniter}
\label{cap:urlr}
%************************************************

Nel capitolo in cui si è introdotto per la prima volta il Controller (vedi sezione\vref{sec:controller}) si è accennato alla gestione degli \ac{URL} in CodeIgniter e a come essi vengono rappresentati.

Lo schema generale adottato da CodeIgniter può essere scomposto nei seguenti elementi:

\sys{miosito.com/[controller]/[metodo]/[argomenti]}

\begin{description}
\item [miosito.com] è l'indirizzo della nostra macchina remota o server locale
\item [controller] è la classe principale che viene invocata per instradare la nostra richiesta
\item [metodo] rappresenta la funzione svolta dal Controller 
\item [argomenti] (facoltativi) sono le variabili che vengono passati al metodo
\end{description}

\section{Gestione degli URL}
Un aspetto peculiare dei moderni framework, a cui CodeIgniter non fa eccezione, è la gestione semplificata degli \ac{URL}, ovvero degli indirizzi che caratterizzano il proprio progetto e che vengono visualizzati nella barra del browser web. Chi ha un po' di dimestichezza con la programmazione \ac{PHP} e la gestione dei parametri, sa bene che molti indirizzi verranno visualizzati nel formato:

\begin{code}
http://www.miosito.com/cartella/val1=pippo&val2=pluto
\end{code}

Dove  ``val1'' e ``val2'' rappresentano due variabili dal valore, rispettivamente di ``pippo'' e ``pluto''. Indirizzi di questo tipo, sono consuetudine nel web, sopratutto nei siti commerciali o nei motori di ricerca (si osservi a tal proposito la barra del browser dopo aver effettuato una ricerca su Amazon o Google ). Gli indirizzi di questo tipo, definiti ``query string'' sono però suscettibili di qualche critica.

\begin{itemize}
\item non sono mnemonici: questo è un dato di fatto
\item vengono visualizzati i nomi delle variabili e i loro valori: questo rende il sito meno impenetrabile. Anche se le comuni regole di sicurezza risolvono questo genere di lacune, sul web è sempre meglio precludere ai malintenzionati il maggior numero di informazioni sensibili
\end{itemize}

In virtù di quanto esposto sopra, CodeIgniter utilizza un sistema di gestione degli \ac{URL} che è nel contempo sia \emph{search-engine friendly} (ottimizzato per l'indicizzazione e il posizionamento delle pagine nei motori di ricerca), sia \emph{human friendly} (semplice da ricordare e digitare per gli utenti). Come funziona questo sistema? Semplice, gli \ac{URL} basati sul sistema querystring vengono riscritti basandosi sulla loro ``segmentazione''. In parole povere, le variabili e i loro parametri diventano meno espliciti all'utente, e ogni \ac{URL} è suddiviso in ``\'' (slash). Questa prassi rende ogni indirizzo più contenuto, facile da ricordare e soprattutto renderà meno gravoso il lavoro SEO per l'indicizzazione del sito nei principali motori di ricerca.

\section{Abilitare le Query String}
Se quanto detto sinora non vi convincesse, potete abilitare la gestione degli \ac{URL} mediante query string, disabilitando quella che CodeIgniter vi propone di default.

\begin{codef}[Esempio di URL basato su query string]
index.php?c=prodottis&m=view&id=345
\end{codef}

\'E sufficiente modificare il file \fil{config.php} e modificare alcune variabili all'interno:

\begin{code}[Esempio di URL basato su query string]
$config['enable_query_strings'] = TRUE; // per abilitare le query string
$config['controller_trigger'] = 'c';
$config['function_trigger'] = 'm';
\end{code}

Nota: utilizzando gli url sotto forma di query string si dovrà prestare attenzione all'utilizzo degli Helper per la generazione degli \ac{URI}, tenendo conto della soluzione adottata.

\section{Aggiungere un suffisso URL}
\'E possibile aggiungere un suffisso alla parte terminale degli \ac{URL} generati, modificando la sezione \var{URL suffix} del file \fil{config.php} che, ricordiamo si trova al percorso \sys{/application/config/} . 

La stringa vuota \verb|$config['url_suffix'] = '';| può essere valorizzata, per esempio, con ``.html'' in modo da cambiare un indirizzo da: 

\verb|tramutare http://mioSito.com/index.php/pages/news/1| 

in:

\verb|http://mioSito.com/index.php/pages/news/1.html|


\section{Un file index di troppo}
Trattando un altro aspetto degli indirizzi, è possibile notare come CodeIgniter inserisca all'interno del path, la stringa \sys{index.php}. Nel caso si desideri eliminare questa informazione dal percorso, si renderà necessario modificare il file \fil{.htaccess}. 

Nell'esempio seguente si utilizza un metodo ``negativo'' con cui ogni cosa viene reindirizzata eccetto gli elementi specificati: index.php, immages e robot.txt.
\begin{code}
RewriteEngine on
RewriteCond $1 !^(index\.php|images|robots\.txt)
RewriteRule ^(.*)$ /index.php/$1 [L]
\end{code}
%
%Successivamente, bisognerà editare nuovamente il file \fil{config.php} in questo modo:
%
%\begin{code}
%$config['index_page'] = "";
%\end{code}
%
%Queste piccole modifiche permettono di nascondere nella barra degli indirizzi, la stringa \sys{index.php}, sempre che questa sia tra le vostre priorità, si intende.