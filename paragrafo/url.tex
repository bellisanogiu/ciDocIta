%************************************************
\chapter{Argomento avanzato I}
\label{cap:url}
%************************************************

Nel capitolo in cui si è introdotto per la prima volta il Controller (si veda la sezione\vref{cap:controller}) si è accennato alla gestione degli \ac{URL} in CodeIgniter e a come essi vengono rappresentati. Nello schema adottato da CodeIgniter possono essere individuati i seguenti elementi:

\begin{code}
miosito.com/[controller]/[metodo]/[argomenti]
\end{code}

\begin{description}
\item [miosito.com] è l'indirizzo della nostra macchina remota o server locale
\item [controller] è la classe principale che viene invocata per instradare la nostra richiesta
\item [metodo] rappresenta la funzione svolta dal Controller 
\item [argomenti] (facoltativi) sono le variabili che vengono passati al metodo
\end{description}

\section{Gestione degli URL}
Un aspetto peculiare dei moderni framework, a cui CodeIgniter non fa eccezione, è la gestione semplificata degli \ac{URL}, ovvero degli indirizzi che caratterizzano il progetto e che vengono visualizzati nella barra del browser web. Chi ha un po' di dimestichezza con la programmazione \ac{PHP} e il passaggio dei parametri, sa bene che molti indirizzi verranno visualizzati nel formato:

\begin{code}
http://www.miosito.com/cartella/val1=pippo&val2=pluto
\end{code}

Dove  ``val1'' e ``val2'' rappresentano due variabili dal valore, rispettivamente di ``pippo'' e ``pluto''. Indirizzi di questo tipo, sono consuetudine nel web, sopratutto nei siti commerciali o nei motori di ricerca (si osservi a tal proposito la barra del browser dopo aver effettuato una ricerca su Amazon o Google). Gli indirizzi di questo tipo, definiti ``query string'' sono però suscettibili di qualche critica.

\begin{itemize}
\item non sono mnemonici: questo è un dato di fatto
\item vengono visualizzati i nomi delle variabili e i loro valori: questo rende il sito meno impenetrabile. Anche se le comuni regole di sicurezza risolvono questo genere di lacune, sul web è sempre meglio precludere ai malintenzionati il maggior numero di informazioni sensibili
\end{itemize}

In virtù di quanto esposto sopra, CodeIgniter utilizza un sistema di gestione degli \ac{URL} che è nel contempo sia \emph{search-engine friendly} (ottimizzato per l'indicizzazione e il posizionamento delle pagine nei motori di ricerca), sia \emph{human friendly} (semplice da ricordare e digitare per gli utenti). Come funziona? Semplice, gli \ac{URL} basati sul sistema query string vengono riscritti basandosi sulla ``segmentazione''. In parole povere, le variabili e i loro parametri diventano meno espliciti all'utente, e ogni porzione dell'\ac{URL} è suddiviso in ``\'' (slash). Questa prassi produce indirizzo dalla lunghezza più contenuta, facili da ricordare e soprattutto renderà meno gravoso il lavoro SEO per l'indicizzazione del sito nei principali motori di ricerca.

\section{Abilitare le Query String}
Se quanto detto sinora non vi convincesse, potete continuare ad usare il sistema query string, disabilitando quello che CodeIgniter vi propone di default.

\begin{codef}[Esempio di URL basato su query string]
index.php?c=prodottis&m=view&id=345
\end{codef}

\'E sufficiente aprire il file \fil{config.php} e modificare alcune variabili all'interno:

\begin{code}[Esempio di URL basato su query string]
$config['enable_query_strings'] = TRUE; // per abilitare le query string
$config['controller_trigger'] = 'c';
$config['function_trigger'] = 'm';
\end{code}

Se si scegli di impostare su TRUE la variabile \verb|enable_query_strings| i controller e i metodi saranno accessibili con le parole \textit{trigger} specificate nel file \fil{config.php}. Di default esse sono rispettivamente \var{c} e \var{m}. Esempio:

\begin{code}
index.php?c=controller&m=method
\end{code}

Nota: utilizzando gli url sotto forma di query string si dovrà prestare attenzione all'utilizzo degli Helper per la generazione degli \ac{URI}, che tengono conto della soluzione adottata.

\section{Aggiungere un suffisso URL}
\'E possibile aggiungere un suffisso alla parte terminale degli \ac{URL} generati, modificando la sezione \var{URL suffix} del file \fil{config.php} che, si ricorda si trova al percorso \sys{/application/config/} . 

La stringa vuota \verb|$config['url_suffix'] = '';| può essere valorizzata, per esempio, con ``.html'' in modo da cambiare un indirizzo da: 

\begin{code}
http://mioSito.com/index.php/pages/news/1
\end{code}

in:

\begin{code}
http://mioSito.com/index.php/pages/news/1.html
\end{code}

\label{sec:index}
\section{Un file index di troppo}
Trattando un altro aspetto degli indirizzi, è possibile notare come CodeIgniter inserisca all'interno del \ac{URL}, la stringa \sys{index.php}. Nel caso si desideri eliminare questa informazione dal percorso, si dovrà modificare il file \fil{.htaccess}. 

Nell'esempio seguente si utilizza un metodo ``negativo'' con cui ogni elemento viene reindirizzato eccetto gli elementi specificati: index.~php, images e robot.~txt.

\begin{code}
RewriteEngine on
RewriteCond $1 !^(index\.php|images|robots\.txt)
RewriteRule ^(.*)$ /index.php/$1 [L]
\end{code}