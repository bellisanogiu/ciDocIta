%************************************************
\section{Evoluzione del progetto}
\label{cap:evoluzione}
%************************************************

In questo capitolo miglioreremo alcuni aspetti del nostro software basandoci su quanto già appreso. Svilupperemo nuovi metodi e renderemo i Controller più articolati e flessibili.

\section*{Metodi multipli in un Controller}
Il nostro Controller \var{Pages} è stato definito, inizialmente con un solo metodo \var{view()} a cui si è aggiunto quello di default \var{index()}. Questo ci fa comprendere che è possibile, anzi generalmente è la prassi, definire più metodi. Esaminiamo lo schema generale di un Controller:

\begin{code}
// Schema generale di un Controller
class Sito extends CI_Controller {
  // definizione della funzione predefinita
  function index() {
    // output della funzione di default 
    echo 'Messaggio stampato dalla funzione predefinita';
  }
  // definizione di una seconda funzione di classe
  function secondafunzione() {
    // output della seconda funzione 
    echo 'Secondo messaggio';
  }
  // definizione di una terza funzione di classe
  function terzafunzione() {
    // output della terza funzione 
    echo 'Terzo messaggio';
  }
}
?>
\end{code}

In questo Controller \var{Sito}, vengono definiti tre metodi, ma nessuno ci vieta di implementarne un numero maggiore. Il primo metodo è \var{index()}: questo è solitamente il nome che viene dato al primo metodo di default. il suo scopo è quello di stampare la stringa \emph{Messaggio stampato dalla funzione predefinita}. Di seguito, i due metodi si occupano di stampare altre stringhe.

Ora creiamo qualcosa che abbia un fine se non nobile, almeno utile per il nostro progetto. Realizziamo un metodo per visualizzare la pagina dei contatti del nostro sito. A questo punto si hanno tutte le conoscenze necessarie per realizzare una pagina statica senza alcun tipo di aiuto: provate a scriverla da soli come pratico esercizio cercando di non sbirciare la prossima sezione.

\section*{La pagina dei contatti}

\begin{code}
<?php

class Pages extends CI_Controller {

	public function index()
	{	$data['title'] = "sito creato con CodeIgniter";
		$data['text'] = "Lista delle notizie principali";
		$data['azienda'] = "realizzata da TizioCaio ";
		
		$this->load->view('header', $data);
		$this->load->view('main', $data);
		$this->load->view('footer', $data);
	}

	public function view($page)
		if ($page!=)
	{
		echo "Ciao! Ti trovi nella sezione $page";
	}
	
	else (se e' uguale a contattaci)
	{
		$data['title'] = "sito creato con CodeIgniter";
		$data['text'] = "Pagina dei contatti";
		$data['azienda'] = "realizzata da TizioCaio ";
			
		$this->load->view('header', $data);
		$this->load->view('contatti', $data);
		$this->load->view('footer', $data);
	}
}
\end{code}

\section*{Aggiungere la logica al controller}
\'E possibile aggiungere facilmente una istruzione condizionale per verificare se una Vista, che intendiamo caricare esiste, o meno. Se la Vista in questione è presente e funzionante, verrà visualizzata, altrimenti si visualizzerà un messaggio di ``avvertimento''.

\begin{code}
public function view($page = 'home')
{

	if ( ! file_exists('application/views/pages/'.$page.'.php'))
	{
		// Ehm, Questa pagina non esiste
		show_404();
	}

	$data['title'] = ucfirst($page); // La prima lettera è resa maiuscola

	$this->load->view('templates/header', $data);
	$this->load->view('pages/'.$page, $data);
	$this->load->view('templates/footer', $data);
}
\end{code}

Il codice si riferisce al Controller \fil{Pages} e al metodo \var{view()} che è stato modificato per effettuare un controllo sul caricamento delle Viste.

\begin{enumerate}
\item \verb|if ( ! file_exists())| \'E una istruzione condizionale del \ac{PHP} che restituisce un valore booleano. Il punto esclamativo trasforma la frase in ``\emph{se non esiste} l'argomento interno
\item \verb|show_404()| Se la View non esiste, questo metodo viene eseguito: si occuperà di visualizzare un messaggio di errore (404 Page not found)
\item \var{ucfirst()} è un metodo che trasforma la prima lettera del valore dato a \verb|$page| (ovvero il titolo) in maiuscolo e lo assegna a \verb|$data['title']|
\item Gli ultimi tre metodi si occupano di caricare le Viste predefinite
\end{enumerate}

In questo modo si sono raggiunti alcuni importanti obiettivi. Si è appreso come verificare l'esistenza o meno di una Vista, e come visualizzare un messaggio di errore automatizzato. Inoltre è lecito aspettarsi che il nome della Vista, sia così esplicativo da poter essere utilizzato anche come titolo della pagina: per questo lo abbiamo assegnato all'indice \var{title} dell'array \var{data}, dopo aver reso maiuscola l'iniziale. Le precedenti Viste create si chiamano \fil{contatti.php}, \fil{main.php} e quindi si prestano bene ad essere utilizzate come titolo.

Con questi piccoli accorgimenti anche la realizzazione delle prossime pagina sarà più rapida.

\section*{Riepilogo}
In questo capitolo abbiamo finalmente aperto il nostro editor per scrivere un po' di codice. Abbiamo migliorato il nostro primo Controller per visualizzare, prima un testo interno e successivamente uno richiamato tramite le View. Questo ci ha fatto apprezzare alcuni degli aspetti del pattern \ac{MVC}: la velocità di sviluppo, il riutilizzo del codice, e la sua manutenibilità elevata. 

La View è quella componente del framework demandata alla visualizzazione delle informazioni. Essa non può essere richiamata direttamente tramite il browser, ma dipende dal Controller. L'utilizzo di questo aspetto cardine del pattern \ac{MVC} fa sì che si realizzi un codice dove la presentazione è separata dalla logica.

\'E stato introdotto anche il passaggio di variabili, ma l'importanza che queste rivestono in un sito dinamico è stata appena messa in evidenza. Chi ha nozioni di programmazione si troverà a suo agio con questi concetti; in fondo, il Controller è una classe che contiene a sua volta uno, o più metodi, mentre il loro argomento rappresenta l'insieme di variabili che andremo ad utilizzare. Il nostro progetto è solo agli inizi, ma già mostra un template suddiviso in header, menu, main e footer, che possono essere modificati indipendentemente. Impareremo nei prossimi capitoli a migliorare quanto abbiamo sviluppato e a introdurre nuovi concetti.