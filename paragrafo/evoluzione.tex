%************************************************
\section{Evoluzione del progetto}
\label{cap:evoluzione}
%************************************************

In questo capitolo miglioreremo alcuni aspetti del nostro software basandoci su quanto già appreso. Svilupperemo nuovi metodi e renderemo i Controller più articolati e flessibili.

\section*{Metodi multipli in un Controller}
Il nostro Controller \var{Pages} è stato definito, inizialmente con un solo metodo \var{view()} a cui si è aggiunto quello di default \var{index()}. Questo ci fa comprendere che è possibile, anzi generalmente è la prassi, definire più metodi. Esaminiamo lo schema generale di un Controller:

\begin{code}
// Schema generale di un Controller
class Sito extends CI_Controller {
  // definizione della funzione predefinita
  function index() {
    // output della funzione di default 
    echo 'Messaggio stampato dalla funzione predefinita';
  }
  // definizione di una seconda funzione di classe
  function secondafunzione() {
    // output della seconda funzione 
    echo 'Secondo messaggio';
  }
  // definizione di una terza funzione di classe
  function terzafunzione() {
    // output della terza funzione 
    echo 'Terzo messaggio';
  }
}
?>
\end{code}

In questo Controller \var{Sito} vengono definiti tre metodi, ma nessuno ci vieta di implementarne un numero maggiore. Il primo metodo è \var{index()}: questo è solitamente il nome che viene dato al primo metodo di default. il suo scopo è quello di stampare la stringa \emph{Messaggio stampato dalla funzione predefinita}. Di seguito, gli altri due metodi si occupano di stampare le relative stringhe.

Ora creiamo qualcosa che abbia un fine se non nobile, almeno utile per il nostro progetto: realizzare un metodo per visualizzare i contatti del nostro sito. A questo punto si hanno tutte le conoscenze necessarie per realizzare una pagina statica senza alcun tipo di aiuto: provate a scriverla da soli come pratico esercizio cercando di non sbirciare la prossima sezione.

\section*{La pagina dei contatti}
Per visualizzare i contatti si è utilizzata una istruzione condizionale \textit{if/else} all'interno del metodo \var{view()}. Il funzionamento è abbastanza intuitivo: se l'argomento passato al metodo è diverso dalla stringa \textit{contatti}, verrà caricata una pagina generica. In caso contrario (ramo else), l'argomento passato è proprio la stringa desiderata, nel qual caso viene caricata la pagina con i riferimenti per contattare gli autori.

\begin{code}
<?php

class Pages extends CI_Controller {

	public function index()
	{	$data['title'] = "sito creato con CodeIgniter";
		$data['text'] = "Lista delle notizie principali";
		$data['azienda'] = "realizzata da TizioCaio ";
		
		$this->load->view('header', $data);
		$this->load->view('main', $data);
		$this->load->view('footer', $data);
	}

	public function view($page)
		if ($page != 'contatti')
	{
		echo "Ciao! Ti trovi nella sezione $page";
	}
	
	else // (se e' uguale a contattaci)
	{
		$data['title'] = "sito creato con CodeIgniter";
		$data['text'] = "Pagina dei contatti";
		$data['azienda'] = "realizzata da TizioCaio ";
			
		$this->load->view('header', $data);
		$this->load->view('contatti', $data);
		$this->load->view('footer', $data);
	}
}
\end{code}

\section*{Aggiungere la logica al controller}
\'E possibile aggiungere facilmente una istruzione condizionale anche per verificare se una Vista, che intendiamo caricare, esiste o meno. Se la Vista in questione è presente e funzionante, verrà visualizzata, altrimenti comparirà un messaggio di ``avvertimento''.

\begin{code}
public function view($page)
{

	if ( ! file_exists('application/views/'.$page.'.php'))
	{
		// Ehm, Questa pagina non esiste
		show_404();
	}

	$data['title'] = ucfirst($page); // La prima lettera è resa maiuscola

	$this->load->view('templates/header', $data);
	$this->load->view('pages/'.$page, $data);
	$this->load->view('templates/footer', $data);
}
\end{code}

Il codice si riferisce al Controller \fil{Pages} e al metodo \var{view()} che è stato modificato per effettuare un controllo sul caricamento delle Viste.

\begin{enumerate}
\item \verb|if ( ! file_exists())| \'E una istruzione condizionale del \ac{PHP} che restituisce un valore booleano. Il punto esclamativo trasforma la frase in ``\emph{se non esiste} l'argomento interno
\item \verb|show_404()| Se la View non esiste, questo metodo viene eseguito visualizzando un messaggio di errore (404 Page not found)
\item \var{ucfirst()} è un metodo che trasforma la prima lettera del valore dato a \verb|$page| (ovvero il titolo) in maiuscolo e lo assegna a \verb|$data['title']|
\item Gli ultimi tre metodi si occupano di caricare le Viste predefinite
\end{enumerate}

In questo modo si sono raggiunti alcuni importanti obiettivi. Si è appreso come verificare l'esistenza o meno di una Vista (e di qualsiasi file), e come visualizzare un messaggio di errore automatizzato. Inoltre è lecito aspettarsi che il nome della Vista, sia così esplicativo da poter essere utilizzato anche come titolo della pagina: per questo lo abbiamo assegnato all'indice \var{title} dell'array \var{data}, dopo aver reso maiuscola l'iniziale. Le precedenti Viste create si chiamano \fil{contatti.php}, \fil{main.php} e quindi si prestano bene ad essere utilizzate come titolo.

Con questi piccoli accorgimenti la realizzazione delle prossime pagine sarà ancora più rapida.