%************************************************
\chapter{Web Page caching}
\label{cap:cache}
%************************************************

Uno degli obiettivi primari che CodeIgniter si prefissa è quello di massimizzare le prestazioni attraverso una serie di operazioni. Una di queste, consente di memorizzare nella cache le pagine prodotte dal proprio progetto. Anche se CodeIgniter è abbastanza veloce, la quantità di informazioni dinamiche visualizzate nelle proprie pagine, unita al numero degli utenti che le producono, possono saturare le risorse del server come la memoria e la potenza di calcolo e quindi influenzare la velocità di caricamento delle pagine. Con il caching\footnote{è una memoria temporanea che memorizza un insieme di dati che possano successivamente essere velocemente recuperati su richiesta.} delle pagine è possibile ottenere delle prestazioni migliori che si avvicinano a quella delle pagine web statiche.

\section{Il funzionamento}
La cache è una memoria temporanea: quando memorizza una pagina dinamica, memorizza tutti i dati in essa contenuti. Il server tira un sospiro di sollievo: non deve interrogare il database, richiedere i valori dinamici, impegnare preziose risorse: gli basta caricare la cache dove è contenuta una ``copia temporanea'' della pagina. Un aspetto importante, è che la modalità caching può essere abilitata e configurata pagina per pagina, decidendo per esempio per quanto tempo una informazione debba essere conservata. Quando una pagina è caricata per la prima volta il file di cache sarà scritto nella cartella \sys{/application/cache/} e le successive richieste non graveranno sul server, ma caricheranno semplicemente il file di cache che verrà inviato al browser del navigatore. Se il file di cache è scaduto (si può impostare un limite temporale per conservarlo) il file di cache verrà cancellato e successivamente aggiornato prima di essere inviato al browser.

Nota: Nota: Il tag ``Benchmark'' non è memorizzato nella cache in modo da poter visualizzare la velocità di caricamento della pagina quando la cache è abilitata.

\section{Abilitare la cache}
Per abilitare il caching è necessario abilitarlo in ogni metodo del proprio controller:

\begin{code}
this->output->cache(n);
\end{code}

Dove il simbolo \var{n} indica i minuti per cui la pagina sarà mantenuta in cache prima che questa venga cancellata e quindi ricaricata. Il codice sopra riportato può andare ovunque all'interno di una funzione. Non è influenzato dall'ordine con cui appare, così che lo si possa inserire dove gli è più congeniale. Una volta che il codice è inserito, le pagine verranno memorizzate nella cache.

Attenzione: A causa del modo con cui CodeIgniter memorizza i contenuti per l'output, il caching funziona solo se si sta generando la visualizzazione per il proprio controller con una Vista. 
Nota: Prima che i file di cache possono essere scritti è necessario impostare i permessi della cartella \sys{/application/cache} in modo ci si possa scrivere.

\section{Cancellare la cache}
Se non si desidera più memorizzare nella cache un file è possibile rimuovere il codice caching e la pagina non sarà più aggiornata quando scade. 
Nota: La rimozione del codice non elimina immediatamente la cache: si dovrà attendere che essa scada normalmente. Se è importante rimuovere la cache immediatamente si dovrà eliminare manualmente il file dalla cartella della cache.