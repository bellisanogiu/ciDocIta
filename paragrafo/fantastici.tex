%************************************************
\chapter{I fantastici Tre}
\label{cap:fantastici}
%************************************************

Gli elementi alla base dei moderni software di sviluppo si basano su tre punti cardine che definiscono per l'appunto l'architectural pattern Model-View-Controller. Originariamente introdotto nel linguaggio Smalltalk, il pattern è stato esplicitamente o implicitamente sposato da numerose tecnologie moderne, come framework basati su PHP (Symfony, Laravel, Zend Framework, CakePHP, Yii framework, CodeIgniter), su Ruby (Ruby on Rails), su Python (Django, TurboGears, Pylons, Web2py, Zope), su Java (Swing, JSF e Struts), su Objective C o su .NET.

Il pattern MVC è in grado di separare la logica di presentazione dei dati dalla logica di business  rendendo lo sviluppo modulare e decisamente più ordinato. In Rete è possibile reperire molta documentazione sull'argomento; tra la tanta, segnaliamo le slide universitarie del Prof.~Maurizio Pizzonia  su \url{http://www.dia.uniroma3.it/~pizzonia/swe/slides/12_MVC.pdf})

\begin{img}{Il pattern MVC}{7}{005}
\end{img}

\begin{img}{I processi}{6}{006}
\end{img}