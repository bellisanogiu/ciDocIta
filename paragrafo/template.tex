%************************************************
\section{Template Engine}
\label{cap:template}
%************************************************

\section*{Introduzione}
La rappresentazione del proprio progetto avviene attraverso le Viste, in cui viene riportato solitamente il codice \ac{PHP}. Questa però non è l'unica e la migliore strada. L'utilizzo di codice di markup per la presentazione, misto a quello demandato a definire la logica, può rendere l'interpretazione delle Viste, caotica e non facilmente gestibile da più sviluppatori contemporaneamente.

CodeIgniter permette di risolvere agevolmente questo problema grazie ad un template engine che rende le Viste più comprensibili senza pregiudicarne le funzionalità. Ma cosa si intende con esattamente con ``template engine''? Con questo termine si intende un sistema che prende in ingresso il linguaggio \ac{PHP} e restituisce un metalinguaggio più semplice da individuare, formattare e scrivere. In pratica viene utilizzata una nuova, più efficiente sintassi per descrivere le funzioni del \ac{PHP} che elimina per esempio tutti gli stati ``echo'' e le parentesi dal codice: questo rende più semplice la cooperazione a tutte quelle persone che, contemporaneamente devono agire sugli stessi file.

Ci sono però degli inconvenienti nell'utilizzo di un nuovo linguaggio come per esempio il dover imparare un nuovo linguaggio per interpretare quello che precedentemente si svolgeva con il \ac{PHP}. Se si intende padroneggiare ogni aspetto di CodeIgniter per utilizzarlo costantemente nei propri progetti futuri, dedicare un po' di tempo alla comprensione del template engine, si rivelerà un investimento redditizio. Bisogna comunque anche tenere conto anche di un altro aspetto, sì purtroppo anch'esso negativo: l'uso di un metalinguaggio penalizza le prestazioni del framework poiché parte del calcolo viene speso per ``interpretare'' il codice da trasformare, in fase di esecuzione, in \ac{PHP}. Quale sia l'impatto di questa scelta sulla reattività del progetto è difficile da quantificare a priori, ma solitamente si tratta di un aspetto trascurabile. In tutti i casi la stima dell'efficienza del progetto con o senza template engine può essere facilmente calcolata con l'uso della ``profilazione'' (si veda \vref{cap:profilazione}).

\section*{Sintassi alternativa}
Il \ac{PHP} permette di abbreviare molti comandi, soprattutto quelli condizionali e interattivi rendendoli più compatti. Non si tratta di una funzione esclusiva di CodeIgniter, ma solo di una sintassi alternativa, che non pregiudica le prestazioni del progetto. Per esempio, il comando ``echo'' per stampare una variabile su schermo ha una sintassi del tipo:

\begin{code}
<?php echo $variable; ?>
\end{code}

Dove ``\verb|<?php|'' e ``\verb|?>|'' racchiudono ogni codice \ac{PHP}, mentre \verb|$variable| è la variabile il cui contenuto sarà stampato su schermo. Ecco una sintassi alternativa per ottenere lo stesso risultato:

\begin{code}
<?=$variable?>
\end{code}

Forse è un po' più ermetica rispetto a quella che utilizzava l'istruzione echo, ma indiscutibilmente è più veloce da scrivere e anche più compatta. Anche le funzioni di controllo come `if`'', ``for'', ``foreach'', ``while'' possono essere semplificate facilmente.

\begin{code}
<?php if ($username == 'Alberto'): ?>

<h3>Ciao Alberto</h3>

<?php elseif ($username == 'zoe'): ?>

<h3>Ciao Zoe</h3>

<?php else: ?>

<h3>Ciao sconosciuto</h3>

<?php endif; ?>
\end{code}

La sintassi alternativa che utilizza gli short tag non prevede parentesi ma utilizza un sistema che delimita il codice in ``blocchi'' ben definiti e facilmente distinguibili. Ogni funzione di controllo prevede un corrispettivo ``tag di chiusura'': endif, endfor, endforeach, e endwhile. Inoltre ognuna di queste strutture, eccetto l'ultima, prevedono nel tag l'uso del carattere ``:'' (due punti). Vediamo come si può riscrivere il ciclo foreach:

\begin{code}
<ul>

<?php foreach ($todo as $item): ?>

<li><?=$item?></li>

<?php endforeach; ?>

</ul>
\end{code}

Se la sintassi si qui definita non funziona come dovrebbe, la soluzione va ricercata in una impostazione presente nel file fil{php.ini} che disabilita gli ``short tag''. In questo caso, il problema è facilmente aggirabile modificando nel framework il file \fil{/config/config.php} con il valore TRUE.

\begin{code}
$config['rewrite_short_tags'] = TRUE;
\end{code}