%************************************************
\chapter{Sicurezza}
\label{cap:sicurezza}
%************************************************


La sicurezza è uno degli argomenti più importanti e spesso più sottovalutati nello sviluppo di un progetto. Per fortuna CodeIgniter mette a disposizione una ricca serie di funzioni che aiutano a curare questo aspetto.

\section{Sicurezza negli URI}

Per minimizzare gli attacchi dovuti a codice malevole inviato al sito web, CodeIgniter è molto restrittivo sul tipo di caratteri che possono essere inclusi nelle stringhe

\begin{code}
testo alfanumerico
tilde (~)
punto (.)
due punti (:)
underscore (_)
dash (-)
\end{code}

\verb|Register_globals|
Durante l'inizializzazione del sistema tutte le variabili globali non sono impostate, eccetto quello contenute negli array \verb|$_GET|, \verb|$_POST| e \verb|$_COOKIE|. Le routine non impostate sono le stesse di \verb|register_globals = off|

\verb|error_reporting|
Nell'ambiente di produzione, è consigliato disabilitare la visualizzazione di qualsiasi tipo di errore o avviso PHP, impostando su 0 il flag interno  \verb|error_reporting|. \'E importante limitare la visualizzazione di informazioni sensibili, in quanto queste potrebbero essere sfruttate per minare la sicurezza del progetto. 

\'E possibile configurare questo aspetto anche tramite CodeIgniter, modificando il file fil{index.php} nella sezione ``application enviroment'' e inserendo uno dei tre valori:

\begin{description}
\item [development] Ambiente di sviluppo. Tutti gli errori e avvisi sono abilitati
\item [testing] Ambiente testing
\item [production] Ambiente di produzione. Tutti gli errori e avvisi sono disabilitati
\end{description}

\verb|magic_quotes_runtime|
Durante l'inizializzazione del sistema la direttiva \verb|magic_quotes_runtime| è impostata su ``off'' in modo da non dover rimuovere il carattere ``slash'' quando si recuperano i valori dal database.

\section{Best practice}
Una raccolta di buone abitudine condivise dagli sviluppatori prende il nome di ``best practice''. Prima di accettare qualsiasi tipo di dati nella propria applicazione, sia che si tratti di dati di tipo POST presi da un form, cookie, URI, XML-RPC, o anche array SERVER, è sempre una buona prassi rispettare nell'ordine queste tre regole:

\begin{enumerate}
\item Non fidarsi. Filtrare i dati come se fossero sempre pericolosi
\item Convalidare i dati\footnote{Spesso il termine utilizzato è validazione dei dati} per assicurarsi che essi rispettino i parametri originariamente impostati come il tipo corretto, la lunghezza, le dimensioni, ecc (a volte questo punto può sostituire quello precedente)
\item Effettuare sempre l'escape dei dati prima di inviarli al database\footnote{Con l'escape dei dati si intende la sostituzione dei caratteri speciali per evitare che il nostro database possa eseguire codice malevolo, invece di semplici dati.}
\end{enumerate}

CodeIgniter fornisce le seguenti funzioni per assistere lo sviluppatore nella sicurezza del proprio software:

\begin{description}
\item[Filtro XSS] CodeIgniter è dotato di un filtro Cross Site Scripting. Questo filtro esamina le tecniche comunemente utilizzate per inserire codice Javascript dannoso o altri tipi di codice che tentano di dirottare i cookie 
\item[Validazione dei dati] Il framework
\item[Escape dei dati] Prima di inviare i dati ottenuti dall'esterno, per esempio tramite un form, è sempre una buona prassi effettuare l'escape dei caratteri nelle query
\end{description}

Il framework CodeIgniter è ricco di funzioni native per la gestione gli errori nella fase di sviluppo. Gran parte del tempo di sviluppo di prodotti complessi viene speso nella fase di debug ed avere degli strumenti efficienti per visualizzarli e gestirli è uno dei punti di successo di un buon framework. 

CodeIgniter mette a disposizione una classe per il logging degli errori che consente di salvare le notifiche di errore e i messaggi prodotti dal debugging all'interno di un comodo file di testo.

Come impostazione predefinita, CodeIgniter mostra gli errori prodotti dalle applicazioni; questa scelta può essere modificata passando il valore "0" alla funzione \verb|error_reporting()| posizionata all'inizio del file \fil{index.php}.
Disabilitare la visualizzazione degli errori non impedirà la scrittura degli stessi all'interno dei file di log.

Le funzioni disponibili per la gestione degli errori sono le seguenti:

\verb|show_error('messaggio' [, int $status_code= 500 ])|

Si tratta di una funzione che ha il compito di permettere la visualizzazione degli errori attraverso un template che si trova al percorso:

\verb|application/errors/error_general.php|

Il parametro \verb|$status_code| non è obbligatorio, ma può essere utilizzato per determinare lo status \ac{HTTP} che verrà associato all'errore inviato al browser.

\verb|show_404('pagina')|

Si tratta di una funzione che presiede alla visualizzazione del messaggio di errore corrispondente allo status 404 ("File not found"); la tipologia di notifica è stabilita attraverso un template che si trova sul seguente percorso

\verb|application/errors/error_404.php|

La funzione produce la notifica di un errore nel caso in cui non sia visualizzabile la pagina che le viene passata come argomento; il framework è comunque impostatato per restituire un messaggio di errore 404 nel caso in cui il tentativo di caricamento di un Controller non abbia successo.

\verb|log_message('livello', 'messaggio')|
Si tratta di una funzione che permette di scrivere dei messaggi all'interno dei logs degli errori; la voce livello prevista come parametro, serve per indicare il tipo di notifica che si desidera scrivere, a questo scopo sono previsti tre livelli: debug, error e info; il seguente esempio chiarirà le differenze tra i tre livelli previsti anche attraverso i commenti:

\begin{code}
// Livelli per la notifica degli errori
$pippo = 0; 
if ($pippo != 0)
{
  // il livello error indica l'errore corrente
  log_message('error', 'Valore inaspettato per la variabile.');
}
else
{
  // il livello debug espone notifiche utili per lo sviluppatore 
  log_message('debug', 'La variabile deve avere valore 0');
}

// il livello info fornisce informazioni su un processo
log_message('info', 'Errore relativo ad una variabile.');
\end{code}

CodeIgniter è stato sviluppato anche con un occhio attento alla sicurezza, per esempio si dimostra altamente restrittivo nei confronti dei caratteri utilizzabili per la generazione delle \ac{URI}, in questo caso infatti è permesso utilizzare nelle stringhe soltanto i caratteri: tilde, punto fermo, due punti, underscore, dash o trattino nonché, naturalmente, i comuni caratteri alfanumerici.

Per quanto riguarda i dati inviati per metodo, è importante sottolineare che \var{GET} viene automaticamente disabilitato quando si utilizzano le \ac{URL} segmentate al posto di quelle basate sulle querystring; ciò avviene perché CodeIgniter impedisce già in fase di inizializzazione la generazione dell'array globale \verb|_GET|; ciò non avviene per quanto riguarda le globali \verb|$_POST| e \verb|$_COOKIE| e in pratica si può dire che il trattamento dei parametri di input da parte del framework simuli del tutto una configurazione di \ac{PHP} con direttiva \verb|register_globals| impostata su OFF.

CodeIgniter è fornito di un filtro nativo contro gli attacchi di tipo \ac{XSS}, esso impedisce quindi in numerosi casi che nelle pagine di un sito venga eseguito del codice Javascript malevolo così come blocca i tentativi di hijacking dei cookie; per utilizzare questo filtro è sufficiente una semplice chiamata:

\begin{code}
// Utilizzo del filtro contro gli XSS
\verb|$data = $this->input->xss_clean($data);|
\end{code}

\begin{code}
application/config/config.php
\end{code}

\section{Introduzione ad .htaccess}
Chi ha avuto modo di amministrare un sito web prima ha avuto prima o poi a che fare \fil{.htaccess}. Si tratta di un comune file dalla grande importanza perché permette di personalizzare le impostazioni globali di accesso del server. Il percorso a cui è disponibile è la directory root del nostro progetto web, ma esso può trovarsi anche in sottodirectory di cui si vuole personalizzare l'accesso.

Le sue funzioni sono innumerevoli e di cruciale importanza nell'ambito della sicurezza. \'E infatti possibile bloccare l'accesso al nostro sito da parte di intere nazioni, numeri ip, definire le pagine di errore o semplicemente gestire l'accesso alle cartelle o ai file secondo determinati criteri. La configurazione del file .htaccess permette in pratica di gestire le richieste in entrata con precisione e con precedenza su qualsiasi altra impostazione definita ad altri software, tra cui il nostro framework CodeIgniter.

\section{Configurare i permessi in entrata}
In CodeIgniter, il file \fil{.htaccess} si trova nella directory application/ e system/ ed è precluso alla visualizzazione tramite il semplice comando ls (il punto che precede il file ha questa funzione). Si scriva pertanto in una delle due cartelle sopra menzionate:

\begin{code}
ls -la
\end{code}

Il quale possiede due argomenti dopo il simbolo del - (trattino):

\begin{description}
\item[a] permette di visualizzare tutti i file, anche quelli nascosti
\item[l] semplicemente visualizza la lista in maniera più ordinata
\end{description}

Ora che si è soddisfatta questa curiosità, possiamo anche modificare i parametri interni (in realtà non è necessario vederlo per aprirlo).

Si apra il file con uno dei comandi seguenti a seconda dell'editor di testo  preferito prestando attenzione ad avere i permessi da superuser:

\begin{description}
\item vi .htaccess
\item gedit .htaccess
\end{description}

CodeIgniter mostrerà la stringa ``Deny from all'' che significa accesso negato per chiunque in entrata.

\section{Bloccare gli indirizzi}
\'E possibile bloccare un solo indirizzo ip o anche un intervallo definito. Il seguente codice commentato è autoesplicativo:

\begin{code}
order allow,deny
deny from 1.2.3.4           # Bloccare un singolo IP
deny from 1.2.3.            # Bloccare tutti IP da 1.2.3.0 a 1.2.3.255
allow from all
\end{code}

\section{Pagina principale}
Solitamente la prima pagina caricata è quella con nome ``index'' ed estensione ``.html''. \'E possibile modificare questa prioritaria, indicando al sistema quale pagina caricare per prima:

\begin{code}
DirectoryIndex NuovoFileIndex.html # Nuova pagina index
\end{code}

Oppure indicare una sequenza di file da caricare, nel caso il precedente non esistesse:

\begin{code}
DirectoryIndex index1.html index2.html index3.html 
\end{code}

\section{Gestire il www del dominio}
\'E possibile caricare il vostro sito rendendo obbligatorio o meno il prefisso \sys{www}

\begin{code}
# Da non www a www
Options +FollowSymlinks
RewriteEngine on
RewriteCond %{HTTP_HOST} ^miosito.it$ 
RewriteRule ^(.*)$ http://www.miosito.it/$1 [R=301,L]
\end{code}

\begin{code}
# Da www a non www
Options +FollowSymlinks
RewriteEngine on
RewriteCond %{HTTP_HOST} ^www.miosito.it$ 
RewriteRule ^(.*)$ http://miosito.it/$1 [R=301,L]
\end{code}

\section{Reindirizazre un URL}

Se nel proprio sito sono stati effettuati degli aggiornamenti come può essere un restyling o migrazione di dominio e ci si ritrova con numerose richieste \ac{HTTP} che vanno a finire in errore 404 allora quello che ti serve sono dei Redirect 301

Se i tuoi errori 404 sono pochi, allora puoi adottare una soluzione facile come ad esempio:

\begin{code}
Redirect 301 /vecchia-pagina.htm http://www.miosito.it/nuova-pagina.htm
\end{code}

Una soluzione migliore, anche se leggermente più complicata, prevede l'utilizzo del \verb|mod_rewrite|. Un esempio concreto può essere: riscrivere tutte le URL con suffisso .html in .php , questo perché abbiamo cambiato il sito statico in  dinamico.

\begin{code}
<ifModule mod_rewrite.c>
RewriteRule ^(.*).Html$ http://www.miosito.com/$1.php[R=301,NC]
</ifModule>
\end{code}

Esempio di reindirizzamento di tutte quelle pagine che hanno all'interno del URL una variabile uguale del tipo \verb|http://www.miosito.it?pagina=4|

\begin{code}
<ifModule mod_rewrite.c>
RewriteCond %{QUERY_STRING} pagina=(.*)$
RewriteRule . http://www.miosito.it/nuova-pagina? [R=301,L]
</ifModule>
\end{code}

Un altro esempio di utilizzo: Se il nostro sito si presenta con indirizzo del tipo:  \verb|http://www.miosito.it/eventi.php?categoria=musica&id=4| si può trasformare in una \ac{URL} seo-friendly del tipo: \verb|http://www.miosito.it/categoria/musica/4|

\begin{code}
<ifModule mod_rewrite.c>
RewriteRule ^categoria/([^/]*)/([^/]*)$ /eventi.php?categoria=$1&id=$2 [L]
</ifModule>
\end{code}

\section{Riepilogo}
\omissis