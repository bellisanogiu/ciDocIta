%************************************************
\chapter{I moderni pattern di sviluppo}
\label{cap:pattern}
%************************************************

Lo scopo di questo capitolo è quello di introdurre ai principi base del framework CodeIgniter e all'architettura \ac{MVC}. Verrà illustrato come una applicazione viene costruita mattone dopo mattone con questo potente e flessibile strumento. Non ci si lasci scoraggiare dalle difficoltà iniziali perché man mano che si procederà nella lettura, molti termini arcani diverranno sempre più familiari. Ad ogni modo, le caratteristiche salienti di CodeIgniter saranno introdotte gradualmente e con profusione di spiegazioni e dettagli al fine di facilitare l'apprendimento. I prossimi capitoli saranno dedicati all'implementazione di una semplice applicazione per visualizzare un elenco di notizie: inizialmente ci si concentrerà su una piattaforma software elementare con semplici pagine statiche e poche informazioni preconfezionate. Successivamente, man mano che si prenderà familiarità con il framework, verranno aggiunte nuovi funzionalità come il caricamento dinamico delle informazioni tramite un database e la loro memorizzazione mediante un pratico form. Verranno esaminati i seguenti argomenti:

\begin{itemize}
\item le basi Model-View-Controller
\item le basi del Routing
\item validazione dei form
\item esecuzione delle query di database usando “Active Record”
\end{itemize}

L'intero tutorial è diviso in più capitoli, ognuno dei quali illustrerà distinte parti delle funzionalità del framework CodeIgniter e del pattern \ac{MVC}.

\section{Separare presentazione e logica}
Questo è un aspetto cardine della moderna programmazione che vogliamo introdurre sin da subito. Converrete che l'ordine, in qualsiasi contesto, è un aspetto positivo, no? In passato questo non era facilmente raggiungibile, in quanto sia il codice logico (adibito alle funzionalità del software) che quello di presentazione (per la rappresentazione grafica) venivano inseriti nello stesso file, con la generazione così di codice caotico, spesso prodotto da più programmatori. Un grafico, per esempio, deve barcamenarsi con porzioni di codice a lui sconosciute: basta un attimo di disattenzione come cancellare inavvertitamente un punto e virgola per introdurre degli errori logici nel codice. E questi, per quanto banali, non sempre risultano facili da scovare. Il programmatore dal suo canto deve gestire un codice in cui ``altri'' mettono le mani (aspetto che già di suo alimenta le sue già notevoli paranoie). 

La pratica di inserire nello stesso file codici dai diversi scopi (di solito presentazione e logica) viene definita come ``inline code''. Un esempio pratico è dato da quei file di codice in cui è presente sia il \ac{PHP} (per interfacciarsi con il database) che l'\ac{HTML} (adibito alla visualizzazione delle pagine web). Oggi per fortuna la situazione sta cambiando velocemente e forse l'esempio più conosciuto di separazione tra codice e presentazione è dato proprio dall'uso congiunto di \ac{HTML} e \ac{CSS}. Il primo viene adibito alla definizione delle informazioni e al loro valore semantic, mentre il secondo definisce la presentazione che verrà mostrata all'utilizzatore finale. Sì, come è lecito supporre da quanto appena detto, il codice \ac{HTML} e quello \ac{CSS} si trovano in due file distinti. Vediamo quali sono i vantaggi di questo approccio.

\begin{description}
\item [riusabilit\'a] \'E possibile usare nuovamente il codice di presentazione in cui sono definiti, per esempio, i colori, il tipo di carattere e la grandezza del font. Basta copiare il file in un nuovo progetto e potrete dire al vostro capo che state sudando ``nuovamente'' le proverbiali sette camicie, mentre impunemente sorseggiate un aperitivo sulla spiaggia. 
\item [ordine] File diversi per scopi diversi, rendono la vita migliore di ognuno di noi. Se volete introdurre una nuova funzionalità nel progetto, non avrete la necessità di toccare i template grafici, giusto?
\item [meno litigi] Non si dovranno più guardare in cagnesco i colleghi che hanno accesso al nostro prezioso codice. Ognuno lavorerà sul proprio file (o quasi) riducendo drasticamente la possibilità di errori difficili da scovare.
\item [rapidit\'a] Le fasi di sviluppo e soprattutto di debug diverranno più rapide.
\end{description}

Separando la presentazione, ovvero la grafica di un software dalla logica che ne espleta le funzionalità, si sono già avuti dei grandi vantaggi, non credete?


\label{sec:config}
\section{Iniziare con CodeIgniter}
Ogni software ha bisogno di un certo impegno per essere padroneggiato. Questa guida vuole minimizzare la curva di apprendimento in modo da rendere il processo di apprendimento di CodeIgniter il più amichevole possibile.

Il primo passo è installare CodeIgniter e seguire i consigli di questa guida (si veda il capitolo\vref{sec:installazione}). Una grande risorsa di informazioni e richieste di aiuto è data dal Forum della Comunità a cui sottoporre domande o problemi \url{http://ellislab.com/codeigniter/}.

\section{Model-View-Controller}
Questo capitolo introduce il paradigma più importante alla base dei framework come CodeIgniter e della sviluppo di software professionale. Non si tratta di concetti difficili da apprendere, ma è bene prestare molta attenzione. Ricordate quando si è parlato di separare la presentazione dalla logica? Bene il discorso del pattern \ac{MVC} ne è solo una rivisitazione ampliata. Lo scopo ultimo è sempre lo stesso, separare i diversi aspetti di un progetto in modo da facilitarne la riusabilità e la manutenzione del codice e, aspetto da non sottovalutare, la collaborazione tra più sviluppatori. Esaminiamo prima di tutte queste tre paroline magiche.

\begin{description}
\item [Model] Il Modello rappresenta la struttura dei dati. Solitamente in esso sono presenti le classi con le funzioni per recuperare, inserire e aggiornare le informazioni sul proprio database.
\item [View] La Vista, come è intuibile dal nome, è dedicata alla rappresentazione grafica delle informazioni. Si desidera formattare il testo con un determinato colore? Ingrandirlo? Bene, il codice che si occuperà dell'aspetto puramente grafico verrà inserito in questa sezione.
\item [Controller] Il Controller (no, ci spiace, non lo tradurremo in controllore) è un nuovo elemento, che ancora non è stato introdotto nel nostro discorso. Esso si occupa, al pari di un router, di instradare una richiesta \ac{HTTP} verso una determinata funzione. Un esempio renderà più chiaro il concetto: ogni volta che si digita un indirizzo nel proprio browser si effettua una richiesta del tipo: ``caro browser, caricami la pagina www.miourl.com''. Il browser analizza la richiesta e se ammissibile (se il sito esiste per esempio), visualizza la pagina desiderata. Questo è il compito di un Controller: analizzare la richiesta e instradarla verso il percorso stabilito. 
\end{description}

Questi tre elementi, trattati per ora molto superficialmente, sono peculiari a quasi tutti i moderni e più evoluti framework web. Padroneggiarne la filosofia è prioritario, ma non ci si preoccupi: con i prossimi capitoli ci si ritroverà ad utilizzarli senza quasi accorgersene.