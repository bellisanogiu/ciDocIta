%************************************************
\chapter{Scripting}
\label{cap:script}
%************************************************

Gli engine dei template non possono eguagliare le performance del \ac{PHP} nativo e la sintassi che si deve imparare per padroneggiarli di solito è solo marginalmente più facile che apprendere le basi del \ac{PHP}. Consideriamo questa porzione di codice:

\begin{code}
<ul> 
<? php foreach ($ rubrica da $ nome):> 
<li><? = $ name></ li> 
<php endforeach; ?> 
</ul>
\end{code}

In contrasto con lo pseudo-codice utilizzato da un engine di template:

\begin{code}
<ul>
{foreach from=$addressbook item="name"}
<li>{$name}</li>
{/foreach}
</ul>
\end{code}

Sì, l'esempio dell'engine di template è leggermente più pulito, ma questo si paga con delle prestazioni minori, poiché lo pseudo-codice deve essere poi convertito in \ac{PHP} per funzionare. Poiché uno dei nostri obiettivi è il massimo delle prestazioni, abbiamo deciso di non rendere obbligatorio l'uso di un engine di template.
\section{Riepilogo}
\omissis