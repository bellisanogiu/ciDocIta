%************************************************
\chapter{Argomento avanzato III}
\label{cap:stile}
%************************************************
\section{Stile e sintassi del PHP}
Qui di seguito verranno descritte le regole generali nello sviluppo del codice con CodeIgniter.

\section*{Formato dei file}
I file dovrebbero essere salvati sempre con l'encoding Unicode UTF-8. Il formato BOM, molto comune negli editor di default di Windows, non dovrebbe mai essere utilizzato: questo formato inserisce dei caratteri nel file che può apportare degli effetti negativi sul codice \ac{PHP}. \'E importante inoltre scegliere come terminatore di riga\footnote{In inglese line ending} l'encoding Unix LF.

Qui di seguito vengono descritti i parametri per configurare due editor testuali molto utilizzati:

\subsection*{Textmate}
\begin{itemize}
\item Aprite le Preferenze dell'Applicazione
\item Selezionate Avanzata e quindi Salvataggio
\item Selezionate UTF-8 (raccomandato) in ``File Encoding''
\item Mentre in ``Line Endings'', si utilizzi la voce Seleziona LF (raccomandato)
\end{itemize}

\subsection*{BBEdit}
Aprite le Preferenze dell'Applicazione
Si selezioni ``Text Encodings'' sulla sinistra
In ``Default text encoding for new documents'' si selezioni ``Unicode (UTF-8, no BOM)''
Si selezioni ``Text Files'' sulla sinistra
In ``Default line breaks'', l'opzione corretta da utilizzare è ``Mac OS X and Unix (LF)''

\section*{Tag di chiusura PHP}
Nel \ac{PHP} il tag di chiusura utilizzato è \verb|?>| anche se spesso non viene riportato, poiché per il parser del linguaggio questo tag è opzionale. Se quindi, contrariamente ad altri linguaggi non lo vedete a fine pagina, non ci si preoccupi: per gli sviluppatori professionisti è normale trascurarlo. Infatti, il problema che può verificarsi è che vengano introdotti dopo il tag di chiusura, eventuali spazi vuoti che possono causare errori \ac{PHP}, pagine vuote, o problemi nelle applicazioni \ac{FTP}. Per questo motivo è consigliato omettere sempre il tag di chiusura e utilizzare invece un blocco di commento per segnare la fine del file e la sua posizione rispetto alla radice dell'applicazione. Ciò consente di marcare il file come completo.

\begin{code}
// esempio non corretto
<?php

echo "Questo è il mio codice non corretto!";

?>

// esempio corretto
<?php

echo "Questo è il mio codice corretto!";

// fine del file myfile.php
// percorso ./system/modules/mymodule/myfile.php
\end{code}

\section*{Nomi delle Classi e dei Metodi}
I nomi delle classi devono incominciare sempre con la prima lettera maiuscola, mentre le parole all'interno del file devono essere separate tramite il carattere underscore, senza utilizzare la convenzione polacca\footnote{Conosciuta anche come CamelCase, è una convenzione con cui i nomi composti da più parole vengono scritti senza spazi, con l'iniziale maiuscola come per esempio ``nomeDelFile''.}

I nomi dei metodi invece devono essere autoesplicativi, chiari e preferibilmente includere un verbo; inoltre devono essere scritti interamente in minuscolo. Cercate di evitare nomi troppo lunghi e dettagliati.

\begin{code}
// Non corretto
class superclass
class SuperClass

// Corretto
class Super_class
\end{code}


\begin{code}
class Super_class {

	function __construct()
	{

	}
}
\end{code}

Qui invece si riportano degli esempi corretti, e meno sulla denominazione dei metodi:

\begin{code}
// Non corretto
function fileproperties()	// non descrittivo, manca l'underscore
function fileProperties()	// non descrittivo e usa la convenzione CamelCase
function getfileproperties()	// meglio, ma continua a mancare l'underscore
function getFileProperties()	// utilizza CamelCase
function get_the_file_properties_from_the_file()	// prolisso

// Corretto
function get_file_properties()	// descrittivo, underscore e lettere minuscole
\end{code}

\section*{Commenti}
In generale il codice dovrebbe essere commentato con cura. Questo perché aiuta i programmatori meno esperti che devono comprendere il nostro lavoro, e noi stessi quando si deve ritornare sul codice a distanza di mesi. Non c'è un formato richiesto per i commenti ma si raccomanda quanto segue.

I commenti in stile DocBlock\footnote{http://phpdoc.org/docs/latest/guides/docblocks.html} precedono la dichiarazione delle classe e dei metodi che possono essere raccolti dagli \ac{IDE}.

\begin{code}
/**
 * Super Class
 *
 * @package	Package Name
 * @subpackage	Subpackage
 * @category	Category
 * @author	Author Name
 * @link	http://example.com
 */
class Super_class {
\end{code}

\begin{code}
/**
 * Encodes string for use in XML
 *
 * @access	public
 * @param	string
 * @return	string
 */
function xml_encode($str)
\end{code}

Utilizzare i commenti a singola linea per brevi spiegazioni, invece lasciare una riga vuota tra grandi blocchi di commento e il codice.

\begin{code}
// break up the string by newlines
$parts = explode("\n", $str);

// A longer comment that needs to give greater detail on what is
// occurring and why can use multiple single-line comments.  Try to
// keep the width reasonable, around 70 characters is the easiest to
// read.  Don't hesitate to link to permanent external resources
// that may provide greater detail:
//
// http://example.com/information_about_something/in_particular/

$parts = $this->foo($parts);
\end{code}

\section*{Le costanti}
Le costanti non differiscono dalla regole di stile definite per le variabili, tranne per il fatto che queste vanno indicate completamente in maiuscolo.

\begin{code}
// Non corretto
myConstant	// manca l'underscore separatore e non è in maiuscolo
N		// sbagliato riportare le costanti con un singolo carattere
S_C_VER		// non descrittivo
$str = str_replace('{foo}', 'bar', $str);	// usare le costanti LD e RD

// Corretto
MY_CONSTANT
NEWLINE
SUPER_CLASS_VERSION
$str = str_replace(LD.'foo'.RD, 'bar', $str);
\end{code}

\section*{TRUE, FALSE, e NULL}
Sono parole chiave che si dovrebbero riportate sempre in maiuscolo.

\begin{code}
// Non corretto
if ($foo == true)
$bar = false;
function foo($bar = null)

// Corretto
if ($foo == TRUE)
$bar = FALSE;
function foo($bar = NULL)
\end{code}

\section*{Operatori Logici}
L'utilizzo di \verb|||| (doppio pipe) è sconsigliato sia perché rende il codice meno intuitivo, sia perché si può confondere con il numero 11: meglio usare \verb|OR|. Invece \verb|&&| è preferibile a AND, ma ambedue sono corretti. Nell'utilizzo dell'operatore di negazione \verb|!| è consigliato farlo precedere e seguire da uno spazio.

\begin{code}
// Non corretto
if ($foo || $bar)
if ($foo AND $bar)  // è corretto ma si raccomanda ti utilizzare &&
if (!$foo)
if (! is_array($foo))

// Corretto
if ($foo OR $bar)
if ($foo && $bar) // raccomandato
if ( ! $foo)
if ( ! is_array($foo))
\end{code}

\section*{Typecasting}
Alcune funzioni \ac{PHP} restituiscono FALSE in caso di errore, ma possono anche avere un valore di ritorno come \verb|""| oppure \verb|0|, che sono valutati come FALSE in molti confronti con operatori logici. Per evitare incongruenze è necessario essere espliciti quando si utilizzano valori di ritorno\footnote{Sono dati resi dall'istruzione return.} in modo che il loro successivo utilizzo all'interno di operatori come AND oppure OR, per esempio, non fornisca ambiguità e quindi un risultato non corretto.

\'E consigliato utilizzare delle verifiche rigorose sul tipo di dati delle proprie variabili, utilizzando \verb|===| e \verb|!==| quando necessario. La differenza rispetto ai consueti operatori di confronto è esplicitata nella tabella\vref{tab:oplogici}

\label{tab:oplogici}
\begin{tabx}{cc}
\toprule
 Operatore  & Descrizione \\ 
 \midrule
==  & uguale  \\ 
=== & identico (cioè con uguale valore e medesimo tipo) \\ 
!= & diverso  \\ 
!== &  diverso per valore, ma del medesimo tipo \\ 
> & maggiore \\ 
>= & maggiore o uguale \\ 
< & minore \\ 
<= & minore o uguale \\ 
\bottomrule 
\end{tabx}
\normalsize

\begin{code}
// Non corretto
// Se 'foo' è all'inizio della stringa, strpos restituità 0 e
// l'istruzione condizionale restituirà TRUE
if (strpos($str, 'foo') == FALSE)

// Corretto
if (strpos($str, 'foo') === FALSE)
\end{code}

\begin{code}
// Non corretto
function build_string($str = "")
{
	if ($str == "")	// Cosa succede se viene passato FALSE o il valore intero 0?
	{

	}
}

// Corretto
function build_string($str = "")
{
	if ($str === "")
	{

	}
}
\end{code}


\subsection*{Conversione in boolean nel PHP}
Quando si converte un valore verso un tipo booleano, i seguenti valori sono considerati FALSE (tutti gli altri TRUE):
\begin{itemize}
	\item FALSE ovviamente
	\item l'intero \verb|0| (zero)
	\item il float \verb|0.0| (zero)
	\item la stringa vuota e la stringa \verb|"0"|
	\item un array con zero elementi
	\item un oggetto con nessuna variabile
	\item il valore speciale NULL
	\item un oggetto SimpleXML creato da tag vuoti
\end{itemize}

Nota: \verb|-1| è considerato TRUE così come altri valori numerici diversi da zero (sia positivi che negativi).

\begin{code}
<?php
var_dump((bool) "");        // bool(false)
var_dump((bool) 1);         // bool(true)
var_dump((bool) -2);        // bool(true)
var_dump((bool) "foo");     // bool(true)
var_dump((bool) 2.3e5);     // bool(true)
var_dump((bool) array(12)); // bool(true)
var_dump((bool) array());   // bool(false)
var_dump((bool) "false");   // bool(true)
?>
\end{code}

Si tratta di un argomento di non secondaria importanza, e per questo si rimanda alla documentazione ufficiale: \url{http://us3.php.net/manual/en/language.types.type-juggling.php#language.types.typecasting} che è molto chiara sull'argomento. Il typecasting presenta molte sfaccettature che possono portare a risultati non desiderati.

\begin{code}
$str = (string) $str; // cast $str come una stringa
\end{code}

\section*{Codice di debug}
Nessuna istruzione utilizzata per il debug può essere lasciata nel codice senza essere opportunamente commentata e quindi resa inefficace. Istruzioni come per esempio \verb|var_dump()|, \verb|print_r()|, \verb|die()|, e \verb|exit()| devono essere sempre commentate alla fine del loro utilizzo o direttamente rimosse.

\begin{code}
// print_r($foo);
\end{code}

\section*{Spazi vuoti nei file}
Nessun spazio dovrebbe precedere il tag di apertura del \ac{PHP}, così come non dovrebbero essercene dopo il tag di chiusura (che in questo linguaggio si può omettere). L'output infatti viene memorizzato (lo spazio è un dato) e CodeIgniter può interpretarlo erroneamente proprio a causa di queste informazioni fuorvianti.

\section*{Compatibilit\'a}
Tutto il codice del proprio progetto deve essere scritto tenendo come punto di riferimento la versione 5.1 o superiore del \ac{PHP}, a meno che questo non sia espressamente indicato nei commenti esplicativi del codice. \'E anche consigliato non utilizzare librerie non-standard\footnote{Le librerie standard sono quelle previste nell'installazione base del \ac{PHP}.} a meno che non sia fornita una alternativa quando esse non sono disponibili.

\section*{Nomi comuni dei file e delle classi}
Quando la classe o il nome del file è una parola di uso comune, o potrebbe molto probabilmente essere già stata utilizzata in un altro script \ac{PHP}, il rischio che ci siano più variabili con lo stesso nome è molto alta. Fornire un prefisso univoco può aiutare a prevenire questo genere di problemi definiti anche collisioni. Siccome gli utenti finali potrebbero utilizzare altri add-on o script \ac{PHP} di terze parti ed introdurre a loro volta delle collisioni, è consigliabile scegliere un prefisso che sia univoco come il proprio identificativo come sviluppatore o società.

\begin{code}
// Non corretto
class Email		pi.email.php
class Xml		ext.xml.php
class Import		mod.import.php

// Corretto
class Pre_email		pi.pre_email.php
class Pre_xml		ext.pre_xml.php
class Pre_import	mod.pre_import.php
\end{code}

\section*{Nome delle tabelle di un database}
Anche nella definizione dello schema di una base di dati può presentarsi il problema delle collisioni. Per questo motivo ogni nome di tabella dovrebbe contenere il prefisso \verb|exp_| seguito da un identificativo univoco legato alla propria identità o società

\begin{code}
// Non corretto
email_addresses		// mancano i due prefissi
pre_email_addresses	// manca il prefisso exp_
exp_email_addresses	// manca il prefisso identificativo

// Corretto
exp_pre_email_addresses
\end{code}

Nota: \'E importante essere consapevoli che MySQL ha un limite di 64 caratteri per i nomi di tabella. Questo non dovrebbe essere un problema in quanto i nomi che supererebbe questo limite sarebberi irragionevolmente lunghi. Ad esempio, il seguente nome della tabella supera i 64 caratteri. 

\begin{code}
exp_pre_email_addresses_of_registered_users_in_seattle_washington
\end{code}

\section*{Un file per classe}
\'E consigliato utilizzare file separati per ciascuna classe utilizzata nel proprio progetto, a meno che le classi non siano strettamente correlate. In CodeIgniter per esempio, il file di classe Database contiene a sua volta più classi, ovvero la classe DB, la classe \verb|DB_Cache|, e il plugin Magpie (a sua volta Magpie contiene le classi Magpie e Snoopy).

\section*{Spazi nel codice}
Per lasciare degli spazi nel codice è sempre consigliato utilizzare il tasto di tabulazione (tab). Questa scelta è sempre preferibile al singolo spazio, ottenuto con la barra spaziatrice, perchè permette di formattare meglio il codice e di rendere evidenti gli spazi al primo colpo d'occhio. L'uso del tab inoltre permette di avere un codice più compatto: lo spazio in byte per memorizzare un tab è uguale a quattro singoli caratteri spazio.

\section*{Interruzione di linea}
I file devono essere salvati utilizzando l'interruzione di linea di Unix\footnote{Definito anche come ritorno a capo, o nuova riga o a capo (in inglese newline, line break o carattere end-of-line / EOL)} che ricordiamo  è un carattere speciale usato per gestire la fine di una riga di testo (e quindi non un vero e proprio carattere visibile sullo schermo). Il nome deriva dal fatto che il carattere successivo dopo di esso viene visualizzato su una nuova riga. Solitamente sono gli editor predefiniti di Windows o quelli più elementari (sempre per Windows) a non supportare il sistema LF.

\section*{Indentazione del codice}
Sull'indentazione delle righe del proprio codice di programmazione esistono molte correnti di pensiero, tutte più o meno valide. Ognuno di noi ha poi un proprio pensiero riguardo alla formattazione del codice: chi preferisce lasciare molti spazi, chi invece ama un testo compatto. In tutti i casi, il codice deve essere scritto non solo per essere efficiente, ma anche facile da leggere per chiunque.

\begin{deftab}{Indentazione}{Rappresenta le regole che disciplinano la formattazione dei blocchi di codice: lo scopo prefissato è rendere chiaro e organizzato il codice sorgente allo sviluppatore che lo ha prodotto e a chi lo consulta per la prima volta.}
\end{deftab}

Tra gli stili utilizzati, prenderemo in considerazione quello conosciuto come lo Stile Allman, che prende il nome da Eric Allman, noto programmatore americano. Questo tipo di indentazione è anche conosciuta come stile BSD poiché è stato adottato dallo stesso Allman per la scrittura di molti programmi per BSD Unix.

\begin{code}
while (a == b)
  {
      doSomething();
      doSomething2();
  }
  finalize();
\end{code}

Come si può notare dall'esempio, le parentesi del ciclo while sono inserite in una propria linea e sono formattate allo stesso livello dell'intestazione del ciclo. Questo stile migliora la leggibilità e rende più facile trovare parentesi graffe corrispondenti ad un'istruzione.

\begin{code}
// Non corretto
function foo($bar) {
	// ...
}

foreach ($arr as $key => $val) {
	// ...
}

if ($foo == $bar) {
	// ...
} else {
	// ...
}

for ($i = 0; $i < 10; $i++)
	{
	for ($j = 0; $j < 10; $j++)
		{
		// ...
		}
	}

// Corretto
function foo($bar)
{
	// ...
}

foreach ($arr as $key => $val)
{
	// ...
}

if ($foo == $bar)
{
	// ...
}
else
{
	// ...
}

for ($i = 0; $i < 10; $i++)
{
	for ($j = 0; $j < 10; $j++)
	{
		// ...
	}
}
\end{code}

\section*{Spazi tra parentesi}
in generale non si dovrebbero utilizzare degli spazi tra le parentesi tonde e quadre. Una eccezione è ammissibile solo in presenza di strutture di controllo \ac{PHP} (dichiarazioni, do-while, elseif, for, foreach, if, switch, while), in modo da distinguerle più facilmente dalle funzioni.

\begin{code}
// Non corretto
$arr[ $foo ] = 'foo';

// Corretto
$arr[$foo] = 'foo'; // nessuno spazio nell'array


// Non corretto
function foo ( $bar )
{

}

// Corretto
function foo($bar) // nessuno spazio intorno alle parentesi
{

}


// Non corretto
foreach( $query->result() as $row )

// Corretto
foreach ($query->result() as $row) // un solo spazio per separare la funzione dall'argomento
\end{code}

\section*{Testo localizzato}
Qualsiasi testo in uscita nel pannello di controllo deve utilizzare i metodi messi a disposizione da CodeIgniter in modo da permettere agevolmente la localizzazione del testo nella propria lingua.

\begin{code}
// Non corretto
return "Invalid Selection";

// Corretto
return $this->lang->line('invalid_selection');
\end{code}

\section*{Metodi privati e variabili}
I metodi e le variabili che sono unicamente accessibili tramite la propria classe come utility ed helper dovrebbero utilizzare un prefisso con il simbolo underscore:

\begin{code}
convert_text() // metodo pubblico
_convert_text()	// metodo privato
\end{code}

\section*{Errori PHP}
Il codice deve essere eseguito senza errori e senza fornire messaggi che segnalano malfunzionamenti (warning). Ad esempio, è una buona pratica non accedere mai ad una variabile (come \verb|$ _POST|) senza prima aver controllato che esista con isset().

Assicurarsi che durante lo sviluppo del proprio progetto, la segnalazione degli errori sia attivata per tutti gli utenti (ALL users), e che \verb|display_errors| sia abilitata nell'ambiente \ac{PHP} utilizzato. È possibile controllare questa impostazione con:

\begin{code}
if (ini_get('display_errors') == 1)
{
	exit "Enabled";
}
\end{code}

Su alcuni server dove \verb|display_errors| è disabilitata e non si ha la possibilità di cambiare questa impostazione nel file \fil{php.ini}, spesso è possibile attivarla con:

\begin{code}
ini_set('display_errors', 1);
\end{code}

Nota: Impostare \verb|display_errors| con \verb|ini_set()| in fase di esecuzione non è identico ad averlo abilitato in ambiente \ac{PHP}. Vale a dire, che non avrà alcun effetto se lo script contiene errori fatali.

\section*{Short Open Tag}
Utilizzare sempre gli \verb|open tag|, nel caso in server non abbia abilitato il supporto ai tag short: \verb|short_open_tag enabled|.

\begin{code}
// Non corretto
<? echo $foo; ?>

<?=$foo?>

// Corretto
<?php echo $foo; ?>
\end{code}

\section*{Una istruzione per linea}
Abituarsi a sviluppare un codice pulito e chiaro, quindi mai combinare più istruzioni in una sola linea.

\begin{code}
// Non corretto
$foo = 'this'; $bar = 'that'; $bat = str_replace($foo, $bar, $bag);

// Corretto
$foo = 'this';
$bar = 'that';
$bat = str_replace($foo, $bar, $bag);
\end{code}

\section*{Stringhe}
Per racchiudere le stringhe si dovrebbero utilizzare sempre i singoli apici, invece delle virgolette (doppi apici). L'unica eccezione in cui sono ammissibili è quando all'interno della stringa si utilizza anche il singolo apice, in modo da evitare l'uso di simboli di escape.

\begin{code}
// Non corretto
"My String"	// usare i singoli apici
"My string $foo" // mancano le parentesi
'SELECT foo FROM bar WHERE baz = \'bag\'' // incomprensibile

// Corretto
'My String'
"My string {$foo}"
"SELECT foo FROM bar WHERE baz = 'bag'"
\end{code}

\section{Query SQL}
Le parole chiave di MySQL devono essere rappresentate in maiuscolo: SELECT, INSERT, UPDATE, WHERE, AS, JOIN, ON, IN, etc. Le query lunghe devono essere sempre spezzate su più righe preferibilmente una per ogni elemento chiave.

\begin{code}
// Non corretto
// le parole chiave sono in minuscono e
// la query è troppo lunga per una singola linea
$query = $this->db->query("select foo, bar, baz, foofoo, foobar as raboof, foobaz from exp_pre_email_addresses
...where foo != 'oof' and baz != 'zab' order by foobaz limit 5, 100");

// Corretto
$query = $this->db->query("SELECT foo, bar, baz, foofoo, foobar AS raboof, foobaz
				FROM exp_pre_email_addresses
				WHERE foo != 'oof'
				AND baz != 'zab'
				ORDER BY foobaz
				LIMIT 5, 100");
\end{code}

\section*{Argomenti di default della funzione}
Quando è possibile, fornire sempre un argomento predefinito nelle funzioni. Questo aiuta a prevenire errori di \ac{PHP} dovuti a chiamate sbagliate e fornisce valori certi di default. Esempio:

\begin{code}
function foo($bar = '', $baz = FALSE)
\end{code}