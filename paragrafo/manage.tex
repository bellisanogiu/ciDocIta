%************************************************
\section{Gestire le applicazioni}
\label{cap:manage}
%************************************************
Per impostazione predefinita CodeIgniter gestisce solo un progetto alla volta definito nel percorso \sys{/application/}. A volte però lo sviluppatore ha la necessità di gestire diversi progetti contemporaneamente con una sola installazione del framework; in questo caso si può optare per rinominare o spostare la cartella \sys{/application/}.

Se si intende rinominare la cartella \sys{/application/}, è necessario aprire il file \fil{index.php} e modificare il seguente parametro, specificando il nome desiderato.

\begin{code}
$application_folder = "application";
\end{code}

\'E possibile spostare la cartella in un differente percorso agendo sempre sul \fil{index.php} e modificando il valore:

\begin{code}
$application_folder = "/Path/to/your/application";
\end{code}

\section*{Pi\'u applicazioni alla volta}
Se si desidera lavorare su più progetti utilizzando una sola installazione di CodeIgniter è necessario organizzare i propri progetti in sotto directory, ognuna delle quali conterrà un singolo progetto. Per esempio, nel caso si vogliano creare due applicazioni denominate ``foo'' e ``bar'', la struttura gerarchica delle proprie cartella sarà la seguente:

\begin{code}
applications/foo/
applications/foo/config/
applications/foo/controllers/
applications/foo/errors/
applications/foo/libraries/
applications/foo/models/
applications/foo/views/
applications/bar/
applications/bar/config/
applications/bar/controllers/
applications/bar/errors/
applications/bar/libraries/
applications/bar/models/
applications/bar/views/
\end{code}

Per selezionare il progetto desiderato basterà, come spiegato precedentemente, agire sul file fil{index.php}. Ogni applicazione ha infatti bisogno del file \fil{index.php} che richiama il progetto associato. Per esempio, per selezionare l'applicazione ``foo'':

\begin{code}
$application_folder = "applications/foo";
\end{code}