%************************************************
\chapter{Argomento avanzato II}
\label{cap:librerie}
%************************************************

\section{Le Librerie}
Per gli anglofili le Libraries sono, come si evince dal nome stesso, delle ``raccolte'' di classi suddivise in categorie omogenee. Ogni Libreria viene in aiuto dello sviluppatore per lo svolgimento dei compiti più ripetitivi e meritevoli di grande attenzione come le classi per la gestione della posta elettronica l'upload e il download dei file, la criptazione o la validazione degli input nei form. Anche in questo caso, il vantaggio per lo sviluppatore, è che non si rende necessario conoscere l'implementazione delle classi, ma semplicemente la loro interfaccia pubblica.

Le Librerie predefinite di CodeIgniter si trovano nel percorso \sys{/system/libraries/} mentre per caricarle nel progetto sarà necessaria una funzione con la sintassi:

\begin{code}
$this->load->library('nome_classe');
\end{code}

La seguente istruzione per esempio carica la libreria adibita alla gestione della posta elettronica:

\begin{code}
$this->load->library('email');
\end{code}

Per caricare più raccolte di classi, si utilizza un array in cui si specificano le Librerie desiderate. Nell'esempio, verranno caricate nello stesso tempo, quelle dedicate alle \var{email} e alle \var{table} (tabelle).

\begin{code}
$this->load->library(array('email', 'table'));
\end{code}

Una volta caricata la Libreria si potranno utilizzare le classi definite al suo interno per i più svariati compiti, secondo una sintassi comune del tipo:

\begin{code}
$this->nome_libreria->nome_funzione('argomento');
\end{code}

Quindi se si volesse inviare una email ad un preciso destinatario, si scriverà:

\begin{code}
$this->email->to('info@tuosito.com');
\end{code}

\begin{itemize}
\item email. \'E il nome della Libreria precedentemente caricata
\item to. \'E il nome della classe demandata ad individuare il destinatario dell'email
\item info@tuosito.com. Ovviamente è l'argomento della nostra classe, in pratica l'indirizzo di colui che riceverà la missiva
\end{itemize}

Qui di seguito alcune delle classi più comuni della Libreria \var{email}, precedute dall'istruzione che, si tenga sempre a mente, deve precaricare tale Libreria.

\begin{code}
// Invio di una email con una Libreria di CodeIgniter

// caricamento della Libreria per la gestione delle email
$this->load->library('email');

// mittente dell'email
$this->email->from('mail@miosito.com', 'Giuseppe');

// destinatario dell'email
$this->email->to('info@tuosito.com');

// indirizzi e-mail del destinatario in copia conoscenza
$this->email->cc('posta@lorosito.com');

// indirizzo e-mail del destinatario in copia conoscenza nascosta
$this->email->bcc('max@boss.com');

// oggetto dell'email
$this->email->subject('Invio email con CodeIgniter');

// testo del messaggio
$this->email->message('Email inviata con la Library Email.');

// invio del messaggio
$this->email->send();
\end{code}

\section*{Creare una Libreria}
Il programmatore esperto, nonostante le numerose Librerie già presenti in CodeIgniter, potrebbe avere la necessità di crearne di nuove, di estendere o sovrascrivere quelle esistenti. Chi lo desiderasse può sviluppare le proprie classi e riutilizzarle in futuri progetti. Per separare le Librerie del framework dalle proprie, si utilizza il percorso \sys{/application/libraries/}, in cui si inseriscono quelle realizzate dall'utente. Esistono alcune convenzioni che è bene tenere presente nella sintassi delle Librerie:

\begin{itemize}
\item nomi dei file. Questa volta, contrariamente a quando avveniva con le View, Controller e Model, i file contenenti le classi devono avere l'iniziale maiuscola. Per esempio \fil{Nome-della-classe.php}
\item le classi. Nel codice le classi saranno definite con la lettera iniziale maiuscola
\item le estensioni. L'estensione del file sarà sempre ``.php''.
\end{itemize}

L'esempio seguente mostra un prototipo della sintassi utilizzata per la definizione di una classe:

\begin{code}
<?php 
// controllo sul percorso della Library
if (!defined('BASEPATH')) exit('No direct script access allowed');  

// dichiarazione della classe
class Nome_classe {
  // funzione di classe
  function funzione() {
    // operazioni ed eventuale valore di ritorno
    ...
	}
}
?>
\end{code}

L'istruzione condizionale \emph{if} effettua una verifica sul percorso della Libreria e in caso di valore booleano ``FALSE'' interrompe l'accesso alle classi ivi definite. Si ricordiamo che la propria Libreria andrà salvata in un percorso differente rispetto a quello previsto per le Librerie di CodeIgniter, ovvero \sys{/application/libraries/}.

Le istruzioni per caricare la propria Libreria e invocare la relativa classe, rispettano la sintassi esaminata precedentemente:

\begin{code}
$this->load->library('nome_classe'); // carica la Libreria
$this->nomeclasse->funzione(); // utilizza il metodo funzione()
\end{code}

Nota: il nome della Libreria, nel nostro caso \verb|nome_classe| può essere riportato tutto in minuscolo oppure con l'iniziale maiuscola: CodeIgniter non discrimina nessuno dei due sistemi. Quindi anche il seguente esempio è corretto.

\begin{code}
$this->load->library('Nome_classe'); // iniziale maiuscola
\end{code}

\section*{Passaggio di parametri}
Come ogni funzione/metodo che si rispetti, è possibile passare dei parametri\vref{sec:passaggio} al metodo della classe quando questa viene ``caricata'', come nell'esempio seguente:

\begin{code}
// Passaggio dinamico di variabili ad una Library personalizzata
// definizione dei parametri sotto forma di array
$parametri = array('tipo' => 'auto', 'colore' => 'giallo');

$this->load->library('nome_classe', $parametri);
\end{code}

Dopo aver definito un array (nel nostro esempio con due indici e relativi valori), si carica la libreria \verb|nome_classe| a cui si passa l'array assegnato alla variabile \verb|$parametri|.

Per utilizzare questo sistema, è però obbligatorio configurare la classe costruttore affinché possa ricevere i dati ``passati'':

\begin{code}
<?php if ( ! defined('BASEPATH')) exit('No direct script access allowed');

class Nome_classe {

    public function __construct($parametri)
    {
        // Qualcosa riferito ai $parametri
    }
}
?>
\end{code}

Gli stessi parametri potranno comunque essere passati direttamente come argomento del metodo:

\begin{code}
// Passaggio dinamico di variabili alla funzione di classe
function funzione($parametri) {
  // operazioni ed eventuale valore di ritorno
  ...
}
\end{code}

Nota: è prevista anche un'ulteriore possibilità: passare i parametri attraverso un file di configurazione. Basta creare un file con lo stesso identico nome della classe contenuta all'interno e memorizzarlo nel percorso \sys{/application/config/}. Questo metodo non è però utilizzabile per il passaggio dinamico di parametri.

\section*{Estensione e sovrascrittura}
\'E possibile aggiungere delle nuove funzionalità alle Librerie di sistema, oppure sovrascriverle completamente, inserendo nel percorso \sys{/application/libraries/} la nostra implementazione con lo stesso ``identico'' nome. Se quindi si crea una personale Libreria per la gestione delle tabelle, basterà inserire il file \var{Table.php} con la definizione delle classi nel percorso \sys{/application/libraries}. In questo modo, sovrascriveremo la Libreria predefinita di CodeIgniter situata invece in \sys{/system/libraries/}.

Riepilogando:

\begin{itemize}
\item \'E possibile creare una nuova, inedita Libreria
\item \'E possibile estendere quelle native
\item \'E possibile rimpiazzare quelle native
\end{itemize}

Nota: le classi del \emph{Database} sono la sola e unica eccezione tra le Librerie di CodeIgniter: esse non possono venire sovrascritte o estese.

\section*{Utilizzare le risorse di Codeigniter}
Per accedere alle risorse native di CodeIgniter con le proprie librerie è necessario utilizzare il metodo \verb|get_instance()| che restituisce un super oggetto. Normalmente ogni funzione di CodeIgniter viene richiamata dal proprio Controller con il costrutto \verb|$this|. Per esempio:

\begin{code}
$this->load->helper('url');
$this->load->library('session');
$this->config->item('base_url');
\end{code}

Il costrutto \verb|$this| funziona però solo all'interno dei propri Controller, Modelli, o Viste. Se invece si desiderano utilizzare le classi native dentro le proprie classi è necessario prima di tutto assegnare l'oggetto di CodeIgniter alla variabile:

\begin{code}
$CI =& get_instance();
\end{code}

Ora si utilizzerà \verb|$CI| al posto della variabile \verb|$this|.

\begin{code}
$CI =& get_instance();

$CI->load->helper('url');
$CI->load->library('session');
$CI->config->item('base_url');
\end{code}

Nota: si sarà notato che la funzione di cui sopra \verb|get_instance()| viene passata per riferimento. Questo è molto importante perché in questo modo si può utilizzare l'originale oggetto di CodeIgniter piuttosto che una sua copia.

\begin{code}
$CI =& get_instance();
\end{code}

\section*{Sovrascrivere le Librerie native}
Rinominando i nomi dei file delle proprie Librerie con quelle di CodeIgniter, si utilizzeranno le classi sviluppate autonomamente al posto di quelle native del framework. Per esempio se si è sviluppata una propria Libreria per la gestione delle ``email'' basterà creare un file di nome \fil{Email.php}, inserirlo nel percorso \sys{/application/libraries/} e dichiararlo con la sintassi:

\begin{code}
class CI_Email {

}
\end{code}

Nota: la maggior parte delle classi native recheranno il prefisso \verb|CI_|.

Per caricare le proprie Librerie basterà utilizzare la funzione consueta per il caricamento:

\begin{code}
$this->load->library('email');
\end{code}

Nota: come detto precedentemente la classe Database costituisce una eccezione: non può essere sostituita dalla propria versione.

\section*{Estendere le Librerie native}
Per i più esigenti si possono introdurre nuove funzionalità non presenti in CodeIgniter: non sovrascrivendo l'intera Libreria nativa, ma semplicemente estendendola. Vi sono però due eccezioni:

\begin{itemize}
\item la dichiarazione della classe deve estendere quella del genitore
\item  il nome della nuova classe e del file che la contiene, devono avere il prefisso \verb|MY_| (questo aspetto è comunque configurabile)
\end{itemize}

Per esempio l'estensione della Libreria \var{Email} si ottiene creando il seguente file al percorso \verb|/application/libraries/MY_Email.php|

\begin{code}
class MY_Email extends CI_Email {

}
\end{code}

Se si rende necessario l'uso di un costruttore, si può estendere quello genitore:

\begin{code}
class MY_Email extends CI_Email {

    public function __construct()
    {
        parent::__construct();
    }
}
\end{code}

\section*{Caricare una sottoclasse}
Per caricare una propria sottoclasse, si dovrà usare una sintassi standard, senza utilizzare il proprio prefisso. Ad esempio, per caricare l'esempio precedente che estende la classe Email, si utilizzerà:

\begin{code}
$this->load->library('email');
\end{code}

Una volta caricata, la variabile di classe sarà disponibile come la classe che si sta estendendo. Nel caso della classe email, per tutte le chiamate si utilizzerà:

\begin{code}
$this->email->some_function();
\end{code}

\section*{Configurare il prefisso}
Se si desidera modificare il prefisso standard previsto da CodeIgniter, basta definire il valore assegnato nel consueto file \fil{config.php} in \sys{/application/config/}.

\begin{code}
$config['subclass_prefix'] = 'MY_';
\end{code}

\section{I Driver}
I Driver sono delle ``speciali'' Librerie che possiedono una classe genitore e un, potenzialmente, infinito numero di classi figlie. La peculiarità di queste ultime è che possono accedere alla classe genitore, ma non a quelle dei ``fratelli''. I Driver, insomma, costituiscono una soluzione elegante per ripartire la gestione delle Librerie in classi discrete.

I Driver si trovano in \sys{/system/libraries/} all'interno di una cartella il cui nome è identico alla classe della Libreria genitore. All'interno di questa cartella vi è una sottocartella denominata \sys{drivers}, che contiene tutte le sottoclassi figlio (si veda la struttura gerarchica\vpageref{lst:driver}).

Per utilizzare un Driver è necessario inizializzarlo con un Controller con la seguente funzione:

\begin{code}
$this->load->driver('class_name'); // class_name è il nome della classe driver da invocare
\end{code}

Se si desidera caricare un Driver di nome ``Alcuni genitori'' si utilizzerà la funzione:

\begin{code}
$this->load->driver('alcuni_genitori'); // alcuni_genitori è il nome della classe driver da invocare
\end{code}

I metodi di una classe saranno successivamente invocati:

\begin{code}
$this->alcuni_genitori->alcuni_metodi();
\end{code}

Le classi figlio possono richiamare questi stessi metodi attraverso la classe genitore, senza però iniziarli:

\begin{code}
$this->alcuni_genitori->figlio_uno->alcuni_metodi();
$this->alcuni_genitori->figlio_due->altro_metodo();
\end{code}

\label{lst:driver}
\section*{Creare un Driver}
Qui di seguito si mostra un esempio della struttura gerarchica del Driver:

\begin{code}[Driver: struttura gerarchica]
/application/libraries/Nome_driver
	Nome_driver.php
	drivers
		Nome_driver_sottoclasse_1.php
		Nome_driver_sottoclasse_2.php
		Nome_driver_sottoclasse_3.php
\end{code}

Nota: per mantenere la compatibilità nella struttura (i nomi sono case-sensitive) è consigliato utilizzare per \verb|Nome_driver| il metodo \var{ucfirst()} che rende l'iniziale maiuscola.

\section*{Riepilogo}
Abbiamo esaminato le Librerie, potenti raccolte di funzioni messe a disposizione da CodeIgniter per gestire molti elementi del nostro progetto come le operazioni di una email o la validazione dei campi di un forum, liberando da queste incombenze il programmatore. I Driver sono anch'esse delle Librerie con delle peculiarità importanti. \'E anche possibile sviluppare le proprie Librerie, aggiungere o ridefinire le classi in modo da personalizzare gli ``attrezzi di lavoro'' con una sola limitazione: la propria fantasia.