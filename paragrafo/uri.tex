%************************************************
\chapter{URI routing e caching}
\label{cap:uri}
%************************************************

Abbiamo già introdotto l'argomento del routing, ovvero dell'instradamento collegato agli \ac{URI} nel capitolo \vref{cap:controller}. In quella occasione abbiamo spiegato come il Controller, analizzando un indirizzo possa poi instradarci verso una sezione del sito, visualizzare un messaggio o interfacciarsi con una base di dati. In tale contesto abbiamo anche visto come CodeIgniter (ma anche tanti altri framework) adottino una struttura degli \ac{URL} segmentata, ovvero separata da slash secondo una sintassi ben definita:

\begin{code}
http://miosito.com/classeController/metodo/argomento
\end{code}

Questa struttura non è obbligatoria e può lasciare spazio alla struttura classica basata sulla visualizzazione delle variabili e dei loro valori, oppure può essere ridefinita a proprio piacimento tramite una procedura chiamata ``rimappatura delle \ac{URI}''. In questo capitolo introdurremo le potenzialità dell'URI Routing, ovvero la relazione che intercorre tra la composizione di un \ac{URI} e le corrispondenti classi e metodi di un Controller.

\section*{Il cuore del Route}
Come CodeIgniter ci ha abituato, esiste un file all'interno della sua struttura in cui è possibile analizzare e definire le regole del routing. Questo file è denominato \fil{routes.php} e si trova al percorso \sys{/application/config/}. All'interno vi è un vettore denominato \verb|$route| che consente di specificare delle regole (routes) personalizzate per l'instradamento attraverso metacaratteri (wildcard) ed espressioni regolari. Insomma, configurando adeguatamente questo file è possibile sfruttare appieno le potenzialità di un framework. Analizziamo alcune delle possibilità offerte dall'opportuno settaggio del file \fil{routes.php}.

\section*{Definire un controller di default}
Quando si sviluppa un progetto consistente, questo probabilmente conterrà anche diversi Controller. \'E importante quindi definire il ``controller di default'', quello che verrà caricato quando un \ac{URI} non è definito, o quando si inserisce l'\ac{URL} radice del proprio sito (miosito.com). Il Controller di default di CodeIgniter è:

\begin{code}
$route['default_controller'] = 'welcome';
\end{code}

Ebbene sì è lui il ``responsabile'' del caloroso messaggio di benvenuto. Possiamo modificare questo parametro inserendo il nome di qualsiasi Controller creato in precedenza, per caricarlo con la priorità massima.

\section*{Rimappare la chiamata dei metodi}
Un'altra possibilità offerta è il rimappamento dei metodi tramite la funzione \verb|_remap()| che si presenta con la sintassi:

\begin{code}
public function _remap()
{
    // Qui ci va il codice
}
\end{code}

Il compito di questa funzione è sovrascrivere la chiamata di un metodo, caricandone un altro. Un po' come redirigere tutto il traffico diretto al punto A, verso un nuovo punto B.

\begin{code}
public function _remap($metodo)
{
    if ($metodo == 'alcunimetodi')
    {
        $this->$metodo2();
    }
    else
    {
        $this->default_metodo();
    }
}
\end{code}

L'esempio appena mostrato contiene il metodo \verb|_remap()| che accetta come argomento \var{\$metodo}. L'istruzione condizione \emph{if} effettua una verifica: se il metodo in questione è uguale a quelli elencati in \var{alcunimetodi} allora verrà caricato \var{metodo2} (sovrascrittura, o override del metodo), altrimenti \emph{else} verrà caricato il \verb|default_metodo|. Questo sistema, se padroneggiato, permette di redirigere le rotte generate da determinati \ac{URI} alla destinazione desiderata, in situazioni particolari o di emergenza.

\'E possibile passare un secondo parametro opzionale al metodo \verb|_remap()|. Nell'esempio che segue, l'array può infatti essere utilizzato in combinazione con la funzione del \ac{PHP} \verb| call_user_func_array| per emulare il funzionamento predefinito di CodeIgniter.

\begin{code}
public function _remap($method, $params = array())
{
    $method = 'process_'.$method;
    if (method_exists($this, $method))
    {
        return call_user_func_array(array($this, $method), $params);
    }
    show_404();
}
\end{code}

\section*{Processare l'output}
CodeIgniter possiede una classe di output che permette di inviare i dati da visualizzare automaticamente ad un browser web. Questa funzione chiamata \verb|_output()| mostra automaticamente l'output finale. Ecco la sintassi:

\begin{code}
public function _output($output)
{
    echo $output;
}
\end{code}

%Please note that your _output() function will receive the data in its finalized state. Benchmark and memory usage data will be rendered, cache files written (if you have caching enabled), and headers will be sent (if you use that feature) before it is handed off to the _output() function.
%
%To have your controller's output cached properly, its _output() method can use:
%if ($this->output->cache_expiration > 0)
%{
%    $this->output->_write_cache($output);
%}
%If you are using this feature the page execution timer and memory usage stats might not be perfectly accurate since they will not take into acccount any further processing you do. For an alternate way to control output before any of the final processing is done, please see the available methods in the Output Class.

\section*{Funzioni private}
In alcune occasioni è necessario che certi metodi siano nascosti all'accesso pubblico, definendo per l'appunto un metodo ``privato''. La sintassi necessaria per rendere un metodo inaccessibile mediante l'\ac{URI} è utilizzare il simbolo \emph{underscore} (\verb|_|) davanti al nome del metodo:

\begin{code}
private function _utility()
{
  // il codice del metodo
}
\end{code}

Se ora si prova a digitare direttamente l'\ac{URI} del metodo, questo non sarà accessibile pubblicamente:

\begin{code}
miosito.com/index.php/blog/_utility/
\end{code}

\section*{Organizzare un grande progetto}
Quando si realizza un progetto di grandi dimensioni, è problematico inserire tutti i Controller, le Viste e i Modelli nelle loro directory principali, senza creare una gerarchia che metta ordine.
\'E possibile, anzi è caldamente consigliato, organizzare il prodotto da sviluppare in sottodirectory omogenee e coerenti, ricordando che nella sintassi dell'\ac{URI} si dovrà tenere conto della eventuale gerarchia, indicando anche le sottodirectory. Per esempio, se si inseriscono i Controller per la gestione delle scarpe di un magazzino, in una directory chiamata \emph{prodotti}, il percorso per richiamare il Controller \var{scarpe} sarà:

\begin{code}
application/controllers/prodotti/scarpe.php
\end{code}

\section*{Classe Costruttore}
Se si intende utilizzare un costruttore nel proprio Controller, è assolutamente necessario inserire al suo interno il codice:

\begin{code}
parent::__construct();
\end{code}

Il motivo è presto detto: il costruttore locale dovrà essere sovrascritto da una classe controller genitore che deve essere richiamata manualmente:

\begin{code}
<?php
class Blog extends CI_Controller {

       public function __construct()
       {
            parent::__construct();
            // Il codice del nostro Costruttore
       }
}
?>
\end{code}

I Costruttori sono utili se si ha la necessità di configurare alcuni valori di default, o eseguire un processo predefinito quando la propria classe viene istanziata. I Costruttori non restituiscono un valore, ma sono utilizzati per svolgere dei lavori predefiniti.

\section{I nomi riservati}
Ebbene sì, se programmate da più di qualche settimana saprete che non tutti i nomi sono utilizzabili per definire variabili, costanti e funzioni nel proprio codice: essi sono riservati a CodeIgniter per definire alcuni elementi chiave del framework. \'E importante non utilizzare i seguenti nomi altrimenti le proprie funzioni sovrascriveranno quelle di default di CodeIgniter, con conseguente difficilmente prevedibili. Qui di seguito proponiamo una lista dei nomi riservati da cui dovrete stare alla larga!

\begin{code}
-------------
| CONTROLLER |
-------------
CI_Base
_ci_initialize
Default
index

----------
| METODI |
----------
is_really_writable()
load_class()
get_config()
config_item()
show_error()
show_404()
log_message()
_exception_handler()
get_instance()

-------------
| VARIABILI |
-------------
$config
$mimes
$lang

------------
| COSTANTI |
------------
ENVIRONMENT
EXT
FCPATH
SELF
BASEPATH
APPPATH
CI_VERSION
FILE_READ_MODE
FILE_WRITE_MODE
DIR_READ_MODE
DIR_WRITE_MODE
FOPEN_READ
FOPEN_READ_WRITE
FOPEN_WRITE_CREATE_DESTRUCTIVE
FOPEN_READ_WRITE_CREATE_DESTRUCTIVE
FOPEN_WRITE_CREATE
FOPEN_READ_WRITE_CREATE
FOPEN_WRITE_CREATE_STRICT
FOPEN_READ_WRITE_CREATE_STRICT
\end{code}

% ?????

A tal proposito, si analizzino i valori predefiniti presenti nel file:

\begin{code}
\verb|$route['default_controller'] = ''welcome'';|
\verb|$route['404_override'] = '';|
\end{code}

Il primo, il \verb|default_controller| come si intuisce dal nome, è il Controller predefinito di CodeIgniter, quello che subito dopo la prima installazione ci fornisce un caloroso benvenuto (si veda la sezione \vref{sec:welcome}). Esso, per essere precisi, indica quale Controller debba essere caricato se l'URI non contiene alcun valore.

Invece il secondo valore ``404 override'' è la regola che interviene quando viene specificato un \ac{URL} che non corrisponde a nessuna regola esistente o corretta.

\begin{code}
// rimappatura con sostituzione di un singolo segmento
\verb|$route['pagina'] = "sito";|
\end{code}

In questo caso la regola di routing impone che il termine``pagina'' sia rimappato nel primo segmento dell'URI tramite la classe ``sito''. 

\begin{code}
// rimappatura di diversi segmenti
\verb|$route['pagina/viaggi'] = "sito/news/10";|
\end{code}

In questo secondo caso un'URI che contiene i segmenti ``pagina/viaggi'' sarà rimappata dalla classe ``sito'' e dal metodo ``news'', mentre l'identificatore sarà impostato su ``10''.

\section{Utilizzo di wildcard ed regex}
Solitamente vi è una corrispondenza uno-a-uno tra la stringa \ac{URL} e la corrispondente classe/metodo che fa capo ad un Controller. \'E possibile in tutti i casi modificare la struttura con cui CodeIgniter mappa ogni \ac{URI}, magari perché si proviene da un altro framework, o solo per il gusto di sperimentare. 

Per apportare le modifiche alle regole di routing si deve modificare il file \fil{config} in \sys{/application/config/} e utilizzare le Regex o le wildcard\footnote{Definito anche metacarattere o carattere jolly o wild character o wildcard character, è un carattere che, all'interno di una stringa, non rappresenta sé stesso bensì un insieme di altri caratteri o sequenze di caratteri.}

\section{Wildcard}
Solitamente un rotta in cui è presente una wildcard ha una forma del tipo:

\begin{code}
$route['prodotto/:num'] = "catalogo/cercaprodotti";
\end{code}

La chiave della variabile \verb|$route| definisce l'\ac{URI} con cui verificare una corrispondenza. Nell'esempio, la chiave è costituita da due segmenti distinti. Il primo è identificato con la stringa prodotto, mentre il secondo rappresenta un wildcard. Si può interpretare come \emph{Se l'URI presenta la parola stringa prodotto (primo segmento) seguito da uno o più numeri (secondo segmento) allora si esegua il metodo cercaprodotti del Controller catalogo}

Per verificare valori alfanumerici si utilizzano i seguenti wildcard:

\begin{itemize}
\item (:num) cercherà una corrispondenza nella parte della stringa che contiene solo numeri

\item (:any) cercherà una corrispondenza nella parte della stringa che contiene ogni carattere
\end{itemize}

Un aspetto da tenere sempre a mente è che le regole di routing vengono eseguite secondo l'ordine in cui sono definite. \'E un aspetto questo da non sottovalutare perché porta soventemente ad errori banali, ma fastidiosi da individuare e correggere. 

Qui di seguito riportiamo alcuni esempi che vi illustreranno la potenza e facilità d'uso di questo sistema:

Tutte gli \ac{URI} che contengono la stringa \var{riviste} saranno instradate verso il Controller \var{blog}:

\begin{code}
$route['riviste'] = "blog";
\end{code}

Tutte gli \ac{URI} che contengono la stringa, questa volta composta, \var{blog/joe} saranno instradate verso il Controller \var{blog}, di cui sarà eseguito il metodo \var{utente} con parametro \var{34}:

\begin{code}
$route['blog/joe'] = "blog/utente/34";
\end{code}

Nell'esempio seguente il nostro \ac{URI} è composto da una stringa, seguita dalla wildcard \var{:any}. Quindi tutti gli \ac{URI} che contengono \var{prodotto} e subito dopo qualsiasi carattere, saranno instradate verso il Controller \var{catalogo}, di cui sarà eseguito il metodo \var{cercaprodotto} con parametro \var{34}:

\begin{code}
$route['prodotto/(:any)'] = "catalogo/prodotto";
\end{code}

Questo esempio è un po' più complesso, ma mostra la flessibilità di CodeIgniter. Ogni \ac{URI} che contiene la stringa prodotto, seguita da un valore numerico (wildcard \var{:num}) viene instradato verso la classe catalogo, di cui viene eseguito il metodo \verb|cerca_prodotto_con_id|. Sin qui dovrebbe essere chiaro, ma è la parte che viene subito dopo a risultare inedita e molto interessante. L'argomento del nostro metodo viene indicato con \verb|$1|, dove il simbolo dollaro indica appunto una variabile: infatti non stiamo indicando un argomento specifico, ma un argomento che assume un valore differente a seconda del numero specificato con la wildcard \var{:num}. 

\begin{code}
$route['prodotti/(:num)'] = "catalogo/cerca_prodotto_id/$1";
\end{code}

In questo modo se specichiamo un \ac{URI} come prodotto/33 CodeIgniter eseguirà il Controller-metodo \verb|catalogo/cerca_prodotto_con_id/33|. In pratica abbiamo definito l'instradamento che si utilizza per cercare/visualizzare un prodotto in un archivio, senza dover specificare il prodotto manualmente! Una bella comodità non credete? 

I più attenti si saranno posti maggiori domande sulla sintassi dell'argomento variabile \verb|$1|. Come è facilmente intuibile, esso si riferisce al primo wildcard, ma nulla ci vieta di specificare \verb|$2| per indicare il secondo wildcard e così via.

\section{Regular Expression}
Comunemente chiamate Regex, o espressioni regolari, in italiano, sono un metodo potente e altamente personalizzabile per definire le regole di routing. Di contro imparare a gestirle pienamente non è facile. \'E possibile utilizzare anche le back-reference, ma in questo caso si deve utilizzare il simbolo del dollaro piuttosto che la sintassi con la doppia barra rovesciata.

\begin{code}
$route['prodotto/([a-z]+)/(\d+)'] = "$1/id_$2";
\end{code}

Nell'esempio precedente, un URI simile a \sys{prodotto/camicia/123} richiamerebbe il Controller \verb|camicia| e la funzione \verb|id_123|. 

Si possono anche mescolare e abbinare caratteri jolly con le espressioni regolari.

\section{Rotte riservate}
Esistono due rotte riservate:

\begin{code}
$route['default_controller'] = 'welcome';
\end{code}

Questa rotta indica quale Controller dovrebbe essere caricato se l'\ac{URI} non contiene alcuno dato: in pratica, si definisce la rotta radice del sito. Nell'esempio precedente viene caricata la classe \var{welcome}.

\'E importante definire una rotta radice, altrimenti, in caso di errore, verrà visualizzata una pagina predefinita con l'errore \emph{404}:

\begin{code}
$route['404_override'] = '';
\end{code}

La rotta di qui sopra indica quale Controller deve essere caricato, se non l'\ac{URI} fornito dall'utente non corrisponde a nessun Controller tra quelli disponibili. Se si configura questa rotta, verrà sovrascritta la pagina di default che gestisce \emph{l'errore 404}. Se invece, non si fornisce alcun valore, verrà eseguita la funzione \verb|show_404()| che si trova nel file \verb|error_404.php| nel percorso \sys{/application/errors/}.

Queste, appena indicate, sono le uniche due rotte riservate. Ed esse devono avere la precedenza su qualsiasi rotta che preveda wildcard o espressioni regolari.

\section{Riepilogo}
\omissis