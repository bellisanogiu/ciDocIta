%************************************************
\section{Alternative PHP Cache}
\label{def:apc}
%************************************************

\section*{Come funziona APC}
Uno degli scopi prefissati dal linguaggio PHP e da chi ovviamente lo utilizza nei propri progetti, è quello di ottimizzare le richieste fatte al server, in modo da migliorare le prestazioni dello stesso verso i terminali host, ovvero gli utenti finali.

Il codice di scripting PHP ogni volta che viene richiamato, non viene immediatamente eseguito, ma inizialmente deve essere analizzato e precompilato in un codice intermedio, quindi ottimizzato e alla fine eseguito. Tutti questi passi, risultano ridondanti quando i contenuti della pagina sono immutati (pagine statiche) e risultano penalizzanti se il server deve soddisfare le richieste di innumerevoli client.

Computo degli ``acceleratori'' come APC è appunto quello di velocizzare i passaggi che precedono l'esecuzione ultima del codice PHP, per garantire al server di preservare parte della potenza computazionale. Esaminiamo i passaggi a cui il codice di scripting PHP è sottoposto, partendo dalla sua lettura dal filesystem ed il suo spostamento in memoria.

\begin{enumerate}
\item Lexing: L'interno del codice PHP viene convertito in token o lexicon
\item Parsing: Durante questa fase, i token vengono elaborati per ricavare delle espressioni significative
\item Compilazione: Le espressioni derivate vengono compilate in opcode
\item Esecuzione: L'opcode viene eseguito per ottenere il risultato finale
\end{enumerate}

L'obiettivo di APC è di saltare i passaggi da 1 a 4, sfruttando un segmento di memoria condivisa che faccia da cache: in questa porzione di memoria i codici operativi generati vengono salvati e quando necessari, copiati nel processo di esecuzione per essere eseguiti.

\ac{APC} è in definitiva un'estensione nativa per PHP che svolge il compito di precompilare, ottimizzare e mantenere in memoria il codice intermedio associato agli script PHP dopo la prima richiesta effettuata ad un file PHP.

\section*{Installazione con Windows}
L'installazione, per chi ha già utilizzato librerie esterne a quelle native del PHP non comporterà alcun problema: avviene semplicemente copiando la DLL precompilata nella directory che contiene le altre estensioni. Infine si abilita l'estensione utilizzando la seguente istruzione nel file \fil{php.ini}:

\begin{code}
extension=php_apc.dll
\end{code}

\section*{Installazione con Linux}
La libreria in questione, sotto sistemi Unix o simili, può avvenire compilando manualmente la libreria con il comando \var{phpsize}. Ecco i passi da compiere:

\begin{itemize}
\item scaricare i sorgenti della libreria dal sito di PECL \url{http://pecl.php.net/package/apc}
\item decomprimere i file scaricati in una directory del server utilizzando i permessi di root
\item a questo sarà necessario compilare la libreria.
\begin{code}
phpize
./configure --enable-apc --enable-mmap
make
cp modules/apc.so /usr/local/lib/php
\end{code}
\item riavviare il server web con l'istruzione:
\begin{code}
service mysqld start
service httpd start
\end{code}
\end{itemize}

\section*{Configurazione di APC}

\begin{itemize}
\item \verb|apc.enabled| abilita/disabilita la libreria APC. Il valore 0 disabilita la libreria.
\item \verb|apc.optimization| specifica il livello di ottimizzazione da applicare al bytecode intermedio generato dallo Zend Engine
\item \verb|apc.ttl| specifica per quanti secondi il codice intermedio rimarrà memorizzato nella cache. Un valore uguale a 0 lascerà in memoria il codice per un tempo molto lungo
\item \verb|apc-filters| è possibile impostare dei filtri per indicare quali file salvare in cache e invece quali tralasciare. I filtri sono definiti in una lista contenente particolari espressioni regolari separate da virgole
\item \verb|apc.file_update_protection| alcuni comandi utilizzati per sostituire i file sul server non eseguono operazioni atomiche (non caricano in un file temporaneo e poi sostituiscono quello precedente): in questo APC può trovarsi ad accedere a file incompleti. Per evitare questo inconveniente si indica il ritardo in secondi dopo i quali si procede con l'operazione di compilazione. Il valore standard è 2, che dovrebbe andar bene quasi sempre, salvo casi in cui ci si trovi ad aver a che fare con sorgenti molto lunghi o server particolarmente lenti.
\end{itemize}

\section*{APC al lavoro}
Ecco i passaggi dettagliati di ciò che fa realmente APC

\begin{itemize}
\item Durante l'inizializzazione del modulo (MINIT), APC utilizza mmap per mappare file in memoria condivisa

\item APC durante la richiesta, rimpiazza \verb|zend_compile_file()| con i propri \verb|my_compile_file()|
\end{itemize}

Particolarmente importante è la funzione svolta da \verb|my_compile_file|:

\begin{itemize}
\item Se APC non riesce a trovare il file nella cache, APC chiamerà \verb|zend_compile_file| per compilare il file

\item Prende l'opcode che viene restituito e lo pone nella memoria condivisa che viene letta/scritta dai vari processi child di Apache

\item Conserva nella memoria condivisa anche classi e function tables

\item Copia l'opcode nella memoria dei child di Apache in modo che Zend possa eseguirlo nella fase di esecuzione
\end{itemize}