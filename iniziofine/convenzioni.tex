% !TEX encoding = UTF-8
% !TEX TS-program = pdflatex
% !TEX root = ../Articolo.tex
% !TEX spellcheck = it-IT

%*******************************************************
% Convenzioni
%*******************************************************
\phantomsection
\pdfbookmark{Sommario}{Sommario}
\clearpage
%section*{Convenzioni}
\chapter{Convenzioni}
Nella guida di stampo tecnico, sono presenti riferimenti a codice inline, directory, percorsi di sistema e variabili. Si è cercato di essere il più possibile chiari adottando le seguenti convenzioni basate su alcune regole e colori:

\begin{enumerate}
\item Nonostante il termine ``funzione'' appartenga alla programmazione procedurale e il ``metodo'' a quella ad oggetti, in questo testo si utilizzano senza distinzione i due termini.
\item Diverse parole sono virgolettate, tra cui comandi, nomi di file, o percorsi di sistema. Se non diversamente specificato, i comandi ``tra virgolette'' devono essere eseguiti senza i relativi doppi apici ``''.
\item Il riferimento a directory di sistema, di CodeIgniter verranno evidenziate con un codice come \sys{application/view/}. I alcuni casi il percorso indicato potrebbe contenere nella parte finale anche il file a cui si fa riferimento come in \sys{application/view/index.php}
\item I riferimenti a variabili, costanti e ai valori loro assegnati saranno evidenziati da espressioni come \var{var = view()}.
\item Quando si vorrà mettere in risalto un nome proprio di un file saranno evidenziate con il colore \fil{index.php}.
\item I listati di codice si è preferito metterli in risalto all'interno di apposite tabelle in questo modo:

\begin{code}
// caricamento dell'Helper per la gestione degli array
$this->load->helper('array');

// definizione di un array
$mioarray = array('nome' => 'giuseppe', 
	'cognome' => 'bellisano', 
	'citta' => 'cagliari'
	);

// restituzione di un valore tramite funzione fornita dall'Helper
echo element('nome', $mioarray);
\end{code}
\end{enumerate}
