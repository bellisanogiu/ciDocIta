\chapter{Acronimi} 
\begin{acronym}
\acro{MySQL}{Oracle MySQL}

{\small è uno dei più utilizzati e conosciuti Relational database management system (RDBMS) composto da un client a riga di comando e un server \par}
%
\acro{MVC}{Model-View-Controll}

{\small è un pattern architetturale molto diffuso nello sviluppo di sistemi software, in particolare nell'ambito della programmazione orientata agli oggetti, in grado di separare la logica di presentazione dei dati dalla logica di business \par}
%
\acro{HTML}[HTML]{Hyper Text Mark-up Language}

{\small è il linguaggio di markup solitamente usato per la formattazione di documenti ipertestuali disponibili nel World Wide Web \par}
%
\acro{URL}{Uniform Resource Locator}

{\small è una sequenza di caratteri che identifica univocamente l'indirizzo di una risorsa in Internet, tipicamente presente su un host server \par}
%
\acro{URI}{Uniform Resource Identifier}

{\small è una stringa che identifica univocamente una risorsa generica che può essere un indirizzo Web, un documento, un'immagine, un file, un servizio, un indirizzo di posta elettronica, ecc. \par}
%
\acro{ORM}{Object-relational mapping}

{\small è una tecnica di programmazione che favorisce l'integrazione di sistemi software aderenti al paradigma della programmazione orientata agli oggetti con sistemi RDBMS \par}
%
\acro{PHP}{PHP: Hypertext Preprocessor}

{\small linguaggio di programmazione interpretato, originariamente concepito per lo sviluppo di pagine web dinamiche \par}
%
\acro{XML-RPC}{eXtensible Markup Language-Remote Procedure Call}

{\small è un protocollo utilizzato in informatica che permette di eseguire delle chiamate a procedure remote (RPC) attraverso la rete Internet. Questo protocollo utilizza lo standard XML per codificare la richiesta che viene trasportata mediante il protocollo HTTP. Nonostante la sua semplicità permette di trasmettere strutture dati complesse, chiederne l'esecuzione ed avere indietro il risultato. \par}
%
\acro{PEAR}{PHP Extension and Application Repository}

{\small è un framework e un sistema di distribuzione per codice scritto in PHP \par}
%
\acro{XSS}{Cross-site scripting}

{\small è una vulnerabilità che affligge siti web dinamici che impiegano un insufficiente controllo dell'input nei form \par}
%
\acro{SMTP}{Simple Mail Transfer Protocol}

{\small è il protocollo standard per la trasmissione via internet di e-mail \par}
%
\acro{FTP}{Simple Mail Transfer Protocol}

{\small è il protocollo standard per la trasmissione via internet di e-mail \par}
%
\acro{GD}{GD Graphics Library}

{\small è una libreria scritta da Thomas Boutell e altri per la manipolazione dinamica di immagini \par}
%
\acro{HTTP}{Hypertext Transfer Protocol}

{\small è un protocollo usato come principale sistema per la trasmissione d'informazioni sul web ovvero in un'architettura tipica client-server \par}
%
\acro{OOP}{Object Oriented Programming}

{\small paradigma di programmazione che permette di definire oggetti software in grado di interagire gli uni con gli altri attraverso lo scambio di messaggi \par}
%
\acro{CAPTCHA}{Completely Automated Public Turing test to tell Computers and Humans Apart}

{\small è un test composto da una o più domande e risposte per determinare se l'utente sia un umano (e non un computer o, più precisamente, un bot). \par}
%
\acro{RSS}{Really Simple Syndication}

{\small è un formato per la distribuzione di contenuti Web basato su XML. \par}
%
\acro{MAMP}{MacOS-Apache-MySQL,P (per PHP, PERL, Python)}

{\small set di programmi liberi comunemente utilizzati per realizzare un sito web locale sul sistema operativo Mac OS X della Apple \par}
%
\acro{LAMP}{Linux/GNU-Apache-MySQL,P (per PHP, PERL, Python)}

{\small set di programmi liberi comunemente utilizzati per realizzare un sito web locale sul sistema operativo Linux/GNU \par}
%
\acro{WAMP}{Windows-Apache-MySQL,P (per PHP, PERL, Python)}

{\small set di programmi liberi comunemente utilizzati per realizzare un sito web locale sul sistema operativo Windows di Microsoft \par}
%
\acro{su}{Super User}

{\small definizione o comando impartito da terminale con cui si acquistano i poteri dell'amministratore, con cui compiere azioni altrimenti precluse agli altri utenti (installazione/rimozione del software, ecc) \par}
%
\acro{CSS}{Cascading Style Sheets}

{\small è un linguaggio usato per definire la formattazione di documenti HTML, XHTML e XML \par}
%
\acro{FTP}{File Transport Protocol}

{\small Il protocollo di trasferimento file, in informatica e nelle telecomunicazioni, è un protocollo per la trasmissione di dati tra host basato su TCP. \par}
%
\acro{ASCII}{American Standard Code for Information Interchange}

{\small \'E un sistema di codifica dei caratteri a 7 bit, comunemente utilizzato nei calcolatori, proposto dall'ingegnere dell'IBM Bob Bemer nel 1961, e successivamente accettato come standard dall'ISO (ISO 646) \par}
%
\acro{APC}{Alternative PHP Cache}

{\small \'E una cache opcode libera ed open source per PHP. Il suo obiettivo è quello di fornire un framework aperto e robusto per il caching e ottimizzare il codice intermedio di PHP. \par}
%
\acro{CC}{Carbon Copy}

{\small \'E una modalità di creazione simultanea di più copie di un documento che prevede l'aggiunta al documento dall'abbreviazione "cc" seguita da un elenco di destinatari, detti appunto destinatari in copia conoscenza. \par}
%
%
\acro{BCC}{blind carbon copy}

{\small Definita anche come copia conoscenza nascosta (Ccn), detta anche copia carbone nascosta, traduzione dell'inglese blind carbon copy (Bcc), è un'intestazione dei messaggi di posta elettronica. I destinatari specificati nel campo Ccn ricevono una copia del messaggio inviato, ma il loro indirizzo viene nascosto agli altri destinatari del messaggio. \par}
%
\acro{DNS}{Domain Name System}

{\small \'E un sistema utilizzato per la risoluzione di nomi dei nodi della rete (in inglese host) in indirizzi IP e viceversa. Il servizio è realizzato tramite un database distribuito, costituito dai server DNS. \par}
%
\acro{CSV}{Comma-Separated Value}
srf

{\small \'E un formato di file basato su file di testo utilizzato per l'importazione ed esportazione di una tabella di dati. Non esiste uno standard vero e proprio. \par}
%
\acro{XML}{eXtensible Markup Language}

{\small \'E un linguaggio di markup che definisce e controlla il significato degli elementi contenuti in un documento o in un testo. \par}
%
\acro{CSRF}{CSRF}

{\small abbreviato CSRF o anche XSRF, è una vulnerabilità a cui sono esposti i siti web dinamici quando sono progettati per ricevere richieste da un client senza meccanismi per controllare se la richiesta è stata inviata intenzionalmente oppure no. Diversamente dal cross-site scripting (XSS), che sfrutta la fiducia di un utente in un particolare sito, il CSRF sfrutta la fiducia di un sito nel browser di un utente. \par}
%
\acro{SQL}{Structured Query Language}

{\small \'E un linguaggio standardizzato per database basati sul modello relazionale (RDBMS).  \par}
%
\acro{IDE}{Integrated Development Environment}

{\small \'E un software che, in fase di sviluppo, aiuta i programmatori nella progettazione e creazione di codice sorgente.  \par}
%
%
\acro{HMAC}{keyed-hash message authentication code}

{\small \'E un codice per l'autenticazione dei messaggi basata su una funzione hash. \par}
\end{acronym}