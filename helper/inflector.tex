%************************************************
\section{Helper Inflector}
\label{helper:inflector}
%************************************************

L'Helper Inflector contiene diverse funzioni che permettono di cambiare le parole nel loro plurale, singolare,con notazione camel case, ecc. Il suo caricamento avviene come di consueto con l'istruzione:

\begin{code}
$this->load->helper('inflector');
\end{code}

Sono quindi disponibili le seguenti funzioni:

\verb|singular()| modifica il plurale delle parole, trasformandola nella forma singolare. Per esempio:

\begin{code}
$word = "dogs";
echo singular($word); // Restituisce "dog"
\end{code}

\verb|plural()| modifica il singolare di una parola, trasformandola nel plurale. Per esempio:

\begin{code}
$word = "dog";
echo plural($word); // Restituisce "dogs"
\end{code}

Per fozare il plurale di una parola che termina con end si utilizzi un secondo argomento impostato sul valore booleano TRUE:

\begin{code}
$word = "pass";
echo plural($word, TRUE); // Restituisce "passes"
\end{code}

\verb|camelize| modifica una stringa di parole separate da spazi o underscore in notazione camel case. Per esempio:

\begin{code}
$word = "my_dog_spot";
echo camelize($word); // Restituisce "myDogSpot"
\end{code}

\verb|underscore()| trasforma più parole separate da spazi trasformando questi ultimi in underscore:

\begin{code}
$word = "my_dog_spot";
echo humanize($word); // Restituisce "My Dog Spot"
\end{code}

\verb|humanize()| prende più parole separate dal carattere underscore e li sostituisce con degli spazi singoli. Inoltre ogni parola verrà formattata con l'iniziale maiuscola. Per esempio:

\begin{code}
$word = "my_dog_spot";
echo humanize($word); // Restituisce "My Dog Spot"
\end{code}