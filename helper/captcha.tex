%************************************************
\section{Helper Captcha}
\label{helper:captcha}
%************************************************

L'Helper Captcha mette a disposizione molte funzionalità per la gestione delle immagini captcha. Il suo caricamento avviene con la funzione:

\begin{code}
$this->load->helper('captcha');
\end{code}

\begin{img}{Captcha}{8}{003}
\end{img}

Le funzioni disponibili sono le seguenti:

\verb|create_captcha($data)| genera il captcha come un'immagine sulla base delle specifiche fornite, quindi restituisce un array di dati associativi sull'immagine.

\begin{code}
[array]
(
  'image' => IMAGE TAG
  'time'	=> TIMESTAMP (in microsecondi)
  'word'	=> CAPTCHA WORD
)
\end{code}

L'``immagine'' è il tag dell'immagine corrente:

\begin{code}
<img src="http://example.com/captcha/12345.jpg" width="140" height="50" />
\end{code}

La parola ``time'' è il nome dell'immagine per cui si utilizza un timestamp in microsecondi (senza l'estensione del file). \'E un numero come questo: 1139612155.3422. Invece ``word'' è la parola che appare nell'immagine captcha che, se non è fornita dalla funzione, sarà una stringa casuale.

%Using the CAPTCHA helper

\section*{Utilizzare l'Helper Captcha}

Una volta caricato l'helper è possibile generare un captcha come nel seguente codice:

\begin{code}
$vals = array(
    'word'	 => 'Parola casuale',
    'img_path'	 => './captcha/',
    'img_url'	 => 'http://example.com/captcha/',
    'font_path'	 => './path/to/fonts/texb.ttf',
    'img_width'	 => '150',
    'img_height' => 30,
    'expiration' => 7200
    );

$cap = create_captcha($vals);
echo $cap['image'];
\end{code}

\begin{itemize}
\item la funzione captcha richiede la libreria GD
\item sono richieste unicamente \verb|img_path| e \verb|img_url|
\item se non viene fornita una ``word'', la funzione genererà una stringa ASCII casuale. Altrimenti si potrebbe usare una libreria per la loro generazione casuale.
\item se non viene specificato un percorso per il font TRUE TYPE, sarà utilizzato quello nativo (e decisamente brutto) della libreria GD
\item la directory ``captcha'' deve avere i permessi di scrittura (solitamente 666 o 777)
\item l'indice ``expiration'' espresso in secondi, indica per quanto tempo l'immagine rimarrà nella cartella captcha prima di essere eliminata. Il valore predefinito è di due ore.
\end{itemize}

\section*{Aggiungere un database}

Affinché la funzione captcha funzioni correttamente, sarà necessario aggiungere le informazioni restituite dalla funzione \verb|create_captcha()| al proprio database. Poi, quando l'utente compilerà il form captcha e invierà i dati, si verificherà se questi sono presenti nel database e che non siano scaduti.

Ecco un prototipo della tabella:

\begin{code}
CREATE TABLE captcha (
 captcha_id bigint(13) unsigned NOT NULL auto_increment,
 captcha_time int(10) unsigned NOT NULL,
 ip_address varchar(16) default '0' NOT NULL,
 word varchar(20) NOT NULL,
 PRIMARY KEY `captcha_id` (`captcha_id`),
 KEY `word` (`word`)
);
\end{code}

Qui vi è un esempio di utilizzo dell'Helper con un database. Nella pagina dove verrà visualizzato il captcha, si avrà un codice simile a questo:

\begin{code}
$this->load->helper('captcha');
$vals = array(
    'img_path'	 => './captcha/',
    'img_url'	 => 'http://example.com/captcha/'
    );

$cap = create_captcha($vals);

$data = array(
    'captcha_time'	=> $cap['time'],
    'ip_address'	=> $this->input->ip_address(),
    'word'	 => $cap['word']
    );

$query = $this->db->insert_string('captcha', $data);
$this->db->query($query);

echo 'Submit the word you see below:';
echo $cap['image'];
echo '<input type="text" name="captcha" value="" />';
\end{code}

Quindi nella pagina che accetta l'invio dei dati, il codice sarà come il seguente:

\begin{code}
// Prima di tutto si cancellano i vecchi capthca
$expiration = time()-7200; // Limite di due ore
$this->db->query("DELETE FROM captcha WHERE captcha_time < ".$expiration);	

// Quindi se esiste un captcha:
$sql = "SELECT COUNT(*) AS count FROM captcha WHERE word = ? AND ip_address = ? AND captcha_time > ?";
$binds = array($_POST['captcha'], $this->input->ip_address(), $expiration);
$query = $this->db->query($sql, $binds);
$row = $query->row();

if ($row->count == 0)
{
    echo "Scrivi la parola che compare nell'immagine";
}
\end{code}