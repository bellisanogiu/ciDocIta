%************************************************
\section{Helper Email}
\label{helper:email}
%************************************************

L'Helper fornisce assistenza quando si lavora con le Email. Per funzionalità più potenti comunque si rimanda alla classe Email\vpageref{class:email}.

L'Helper si carica con l'istruzione:

\begin{code}
$this->load->helper('email');
\end{code}

\verb|valid_email('email')| controlla se una email è correttamente formattata, ma non fornisce alcuna garanzia o prova che l'email sarà ricevuta dal destinatario. Il valore restituito può essere TRUE/FALSE.

\begin{code}
$this->load->helper('email');

if (valid_email('email@somesite.com'))
{
    echo 'email valida';
}
else
{
    echo 'email non valida';
}
\end{code}

\verb|send_email('recipient', 'subject', 'message')| invia una email utilizzando la funzione nativa PHP \var{mail()}. Per funzionalità più potenti comunque si rimanda alla classe Email\vpageref{class:email}.