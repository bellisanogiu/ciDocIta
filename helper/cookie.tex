%************************************************
\section{Helper Cookie}
\label{helper:cookie}
%************************************************

L'Helper viene caricato con il seguente codice:

\begin{code}
$this->load->helper('cookie');
\end{code}

Sono tre le funzioni messe a disposizione dall'Helper che assistono lo sviluppatore nel lavoro con i cookie:

\begin{itemize}
\item \verb|set_cookie()| questa funzione fornisce alla propria Vista una sintassi intuitiva per l'impostazione dei cookie del browser. Per maggiori informazioni, si faccia riferimento alla classe Input\vpageref{class:input}.

\item \verb|get_cookie()| fornisce alla propria Vista una sintassi intuitiva per recuperare i cookie del browser. Per maggiori informazioni, si faccia riferimento alla classe Input\vpageref{class:input}.

\item \verb|delete_cookie()| cancella un cookie specificando il solo nome:

\begin{code}
delete_cookie("name");
\end{code}

Questa funzione è identica a \verb|set_cookie()|, tranne che non è caratterizzata da un valore e da parametri di scadenza. \'E possibile inviare un array di valori nel primo parametro oppure si possono impostare i singoli parametri.

\begin{code}
delete_cookie($name, $domain, $path, $prefix)
\end{code}
\end{itemize}