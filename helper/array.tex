%************************************************
\section{Helper Array}
\label{helper:array}
%************************************************

Le funzionalità di questo Helper aiutano nella gestione degli array. Il suo caricamento avviene con l'istruzione:

\begin{code}
$this->load->helper('array');
\end{code}

Qui di seguito si mettono in evidenza le funzioni disponibili:
\begin{itemize}
\item \verb|element()| consente di recuperare un elemento da un array. La funzione verifica se l'indice dell'array esiste e se questo ha un valore. Se esiste un valore, questo viene restituito, se invece il valore non esiste viene restituito FALSE, o qualsiasi altra cosa si sia specificato come valore di default tramite il terzo parametro. Per esempio:

\begin{code}
$array = array('colore' => 'rosso', 'forma' => 'circolare', 'dimensione' => '');

// restituisce "rosso"
echo element('colore', $array);

// restituisce NULL
echo element('dimensione', $array, NULL);
\end{code}

\item \verb|random_element()| viene preso un array in ingresso e restituito un elemento casuale. Un esempio è il seguente:

\begin{code}
$quotes = array(
            "I find that the harder I work, the more luck I seem to have. - Thomas Jefferson",
            "Don't stay in bed, unless you can make money in bed. - George Burns",
            "We didn't lose the game; we just ran out of time. - Vince Lombardi",
            "If everything seems under control, you're not going fast enough. - Mario Andretti",
            "Reality is merely an illusion, albeit a very persistent one. - Albert Einstein",
            "Chance favors the prepared mind - Louis Pasteur"
            );

echo random_element($quotes);
\end{code}

\item \verb|elements()| consente di ottenere il numero di elementi da un array. La funzionalità verifica se ciascuno degli indici dell'array è impostato: se un indice non esiste, questo assume il valoresu FALSE, o qualsiasi altra cosa si sia specificato come valore di default tramite il terzo parametro. Per esempio:

\begin{code}
$array = array(
    'colore' => 'rosso',
    'forma' => 'circolare',
    'raggio' => '10',
    'diametro' => '20'
);

$my_shape = elements(array('colore', 'forma', 'altezza'), $array);
\end{code}

Il codice restituisce il seguente array:

\begin{code}
array(
    'color' => 'rosso',
    'shape' => 'circolare',
    'height' => FALSE // questo indice non esiste
);
\end{code}

\'E possibile impostare il terzo parametro con qualsiasi valore di default desiderato:

\begin{code}
$my_shape = elements(array('colore', 'forma', 'altezza'), $array, NULL);
\end{code}

L'esempio precedente restituisce il seguente array:

\begin{code}
array(
    'color' => 'rosso',
    'shape' => 'circolare',
    'altezza' => NULL // nuovo valore impostato
);
\end{code}

Questo è utile quando si invia l'array \verb|$_POST| ad uno dei propri Model poiché impedisce agli utenti di inviare ulteriori dati POST per essere inseriti nelle proprie tabelle:

\begin{code}
$this->load->model('post_model');

$this->post_model->update(elements(array('id', 'title', 'content'), $_POST));
\end{code}

In questo modo, solo i campi id, title e content saranno aggiornati.
\end{itemize}