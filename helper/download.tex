%************************************************
\section{Helper Download}
\label{helper:download}
%************************************************
L'Helper che fornisce di scaricare i dati sul proprio desktop, viene caricato con l'istruzione:

\begin{code}
$this->load->helper('download');
\end{code}

\verb|force_download('filename', 'data')| genera i server header che forza il download dei dati sul proprio desktop. Questa funzione è utilizzata per scaricare i file sul pc locale. Il primo parametro è il nome che si vuole dare al file scaricato, mentre il secondo parametro è il nome del file da scaricare. Per esempio:

\begin{code}
$data = 'qui inserisci il testo';
$name = 'mytext.txt';

force_download($name, $data);
\end{code}

Se si vuole scaricare dal server un file esistente, sarà necessario leggere il file in una stringa:

\begin{code}
$data = file_get_contents("/path/to/photo.jpg"); // Read the file's contents
$name = 'myphoto.jpg';

force_download($name, $data);
\end{code}