%************************************************
\section{Helper Directory}
\label{helper:directory}
%************************************************
Questo Helper, che offre dei metodi per lavorare più agilmente con le directory, è caricato mediante il codice:

\begin{code}
$this->load->helper('directory');
\end{code}

Qui di seguito un elenco di funzioni disponibili:

\verb|directory_map('source directory')| questa funzione legge il percorso della directory specificata nel primo parametro e produce una array di tutti i file contenuti in essa e nelle sottodirectory. Per esempio:

\begin{code}
$map = directory_map('./mydirectory/');
\end{code}

Nota: il percorso deve essere sempre definito relativamente al file \fil{index.php} nella directory root.

\'E possibile gestire il livello di profondità con cui saranno mappati i file nella directory attraverso un secondo parametro. Il suo valore specificato nel formato intero rappresenterà la profondità di mappatura. Per esempio, impostando il secondo parametro su ``1'' verrà mappata solo la directory definita nel primo parametro.

\begin{code}
$map = directory_map('./mydirectory/', 1);
\end{code}

Per impostazione predefinita, i file nascosti non saranno inclusi nell'array, a menoché non si imposti un terzo parametro con il valore booleano TRUE.

\begin{code}
$map = directory_map('./mydirectory/', FALSE, TRUE);
\end{code}

Ogni nome di directory verrà definito come indice dell'array e i suoi file saranno elencati in ordine numerico. Qui di seguito un esempio di un tipico array:

\begin{code}
Array
(
   [libraries] => Array
   (
       [0] => benchmark.html
       [1] => config.html
       [database] => Array
       (
             [0] => active_record.html
             [1] => binds.html
             [2] => configuration.html
             [3] => connecting.html
             [4] => examples.html
             [5] => fields.html
             [6] => index.html
             [7] => queries.html
        )
       [2] => email.html
       [3] => file_uploading.html
       [4] => image_lib.html
       [5] => input.html
       [6] => language.html
       [7] => loader.html
       [8] => pagination.html
       [9] => uri.html
)
\end{code}