%************************************************
\section{Helper Security}
\label{helper:security}
%************************************************
L'Helper viene caricato con l'istruzione:

\begin{code}
$this->load->helper('security');
\end{code}

Qui di seguito le funzioni disponibili:

\verb|xss_clean()| fornisce un filtro Cross Site Script Hack. Questa funzione è un alias della funzione definita nella classe Input (si veda la sezione\vref{class:input}.

\verb|sanitize_filename()| fornisce una protezione contro il directory traversal. Questa funzione è un alias della funzione definita nella classe Security (si veda la sezione\vref{class:sicurezza}. 

\verb|do_hash()| permette di creare un hash SHA1 oppure MD5 sfruttabile per il criptaggio delle password. Per impostazione predefinita viene utilizzato l'hash SHA1:

\begin{code}
$str = do_hash($str); // SHA1

$str = do_hash($str, 'md5'); // MD5
\end{code}

Nota: questa funzione era precedentemente chiamata \verb|dohash()| ed è stata deprecata a favore dell'attuale \verb|do_hash()|.

\verb|strip_image_tags()| questa funzione di sicurezza priverà una stringa dai tag immagine. Im questo modo il risultato sarà un semplice testo con l'URL dell'immagine.

\begin{code}
$string = strip_image_tags($string);
\end{code}

\verb|encode_php_tags()| è una funzione di sicurezza che converte i tag PHP in entità. Questa funzione viene utilizzata automaticamente se si usa la funzione che permette il filtro XSS:

\begin{code}
$string = encode_php_tags($string);
\end{code}