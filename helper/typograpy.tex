%************************************************
\section{Helper Typography}
\label{helper:typography}
%************************************************

L'Helper viene caricato con il seguente codice:

\begin{code}
$this->load->helper('typography');
\end{code}

\begin{itemize}
\item \verb|auto_typography()| formatta il testo in modo che sia semanticamente e tipograficamente in \ac{HTML}. Si veda la classe Typography\vref{class:typography} per maggiori informazioni. Per esempio:

 \begin{code}
$string = auto_typography($string);
\end{code}

L'uso di questa funzione può essere fonte di un intenso lavoro da parte del server, soprattutto se la quantità di dati da formattare è elevata. Se si utilizza questa funzione, è consigliabile effettuare il caching delle pagine (si veda la sezione\vref{sec:cache}.

\item \verb|nl2br_except_pre()| converte i ritorni a capo nel tag <br /> a meno che questi non siano all'interno dei tag <pre>. Questa funzione è simile a quella nativa PHP \var{nl2br()} tranne per il fatto che questa ignora i tag <pre>.

 \begin{code}
$string = nl2br_except_pre($string);
\end{code}
\end{itemize}